%%% SETUP

\documentclass[slides]{beamer}

\setbeameroption{hide notes}
%\setbeameroption{show notes}
%\setbeamertemplate{note page}[plain]
%\setbeameroption{show notes on second screen}

% Theme:

\usetheme{Malmoe}
\usecolortheme[RGB={101, 56, 25}]{structure}

\setbeamertemplate{itemize items}[default]
\setbeamertemplate{itemize subitem}[circle]

\setbeamertemplate{navigation symbols}{}

% Packages:
\usepackage{amssymb,amsmath,amsfonts,amsthm}
\usepackage{graphicx, tikz}
\usepackage{todonotes}
\usepackage{xkeyval}
\presetkeys{todonotes}{inline}{}

% Commands:
\newcommand{\E}{\text{E}}
\newcommand{\V}{\mathrm{V}}
\newcommand{\C}{\mathrm{Cov}}
\newcommand{\dd}{\mathrm{d}}
\newcommand{\beq}{\begin{equation}}
\newcommand{\eeq}{\end{equation}}

\newcommand{\bit}{\begin{itemize}}
\newcommand{\eit}{\end{itemize}}

\newcommand{\benu}{\begin{enumerate}}
\newcommand{\eenu}{\end{enumerate}}

\definecolor{themebrown}{RGB}{101, 56, 25}

% Title:

\title[How Efficient is the Market for Physician Referrals? \hspace{0.5cm} \insertframenumber]{How Efficient is the Market for \\ Physician Referrals? \\ \ \\ PRELIMINARY}
\author[McCarthy \and Richards-Shubik]%
{Ian McCarthy\inst{1} \and Seth Richards-Shubik\inst{2}}
\institute[]{
  \inst{1} Emory University and NBER
  \and
  \inst{2} Lehigh University and NBER
}
\date[]{Tinbergen Institute \\ \small{December 2022}}

\begin{document}

\frame{\titlepage}

%%% INTRO

\begin{frame}{Physician referrals and informational frictions}
\begin{itemize}

\item Production network (micro, one service) with substantial informational frictions and heterogeneity in prices and quality
\bit
    \item Prices do not clear market in the short run
%    \item Relevant models?
\eit
\medskip \pause

\item Referral networks are concentrated and exhibit homophily \\
{\small (Agha \emph{et al.}~2022; Zeltzer 2020)}
\medskip

\item Limited market response to quality (and price) differences \\
{\small (Kolstad and Chernew 2009; Chandra \emph{et al.}~2016; Gaynor, Propper, and Seiler 2016, Chernew \emph{et al.}~2021)}
\medskip

\item Structural empirical models of physician learning \\
{\small (Crawford and Shum 2005; Ferreyra and Kosenok 2011; Chan, Narasimhan, and Xie 2013; Dickstein 2018; Gong 2018)}

\end{itemize}
\end{frame}

%%%

\begin{frame}{Assessing the (in)efficiency}
\begin{itemize}

%\item Physician practice styles / supply-side factors are a large source of inefficient variations in health care
% variations *within* areas

\item We develop a structural learning model for referral decisions and apply it to a specific area of health care with a well defined network of referring and receiving physicians
\bit
    \item How well do referring physicians learn about specialist quality and direct patients toward better specialists?
    \item What is the scope to improve outcomes and reduce costs via better referrals?
\eit
\bigskip \pause

\item Outline:
\bit
    \item Background on application and data
    \item Description of referral network
    \item Bayesian learning model
    \item Identification (estimation is in process)
    \item Multinomial logit estimates: \\
    1 additional failure reduces referral probability 7\% (relative)
\eit

\end{itemize}
\end{frame}



%%%%%%%%%%%%%%%%%%%%%%%%%%%%%%%%%%%%
% BACKGROUND
%%%%%%%%%%%%%%%%%%%%%%%%%%%%%%%%%%%%

\begin{frame}
\begin{center}
\LARGE
\color[RGB]{89, 38, 11}
BACKGROUND
\end{center}
\end{frame}

%%%

\begin{frame}{Application and data}
\begin{itemize}

\item Major joint replacements
\bit
    \item Expensive: \$20,000 mean/median (\$12,500 std.~dev.) 
    \item Some risk: 9.1\% rehospitalized, 0.6\% die from complications
\eit
\medskip

\item Primary care physicians (PCPs/GPs) refer patients to orthopedic surgeons (specialists) for the procedure
\bigskip \pause

\item US public health insurance (Medicare) data, 2008--2018 \\
% Age 65 and over
% transaction-level data
1,650,000 surgeries, 60,000 PCPs, 17,000 specialists
\medskip

\item Referring physician (PCP) inferred from regular office visits over prior year, or recorded directly on the surgery claim
\medskip

\item Restrict to cases where PCP and specialist have sufficient patient volume %at least 20 total patients in sample and makes referrals in at least 3 consecutive years

\end{itemize}
\end{frame}

%%%

\begin{frame}{Key summary statistics}
\footnotesize

\begin{tabular}{llrrr}
\hline
&    & (Baseline)     & (Estimation) \\
&    & 2008-2012     & 2013-2018 & Overall\\
\hline
\multicolumn{4}{l}{\emph{Per PCP per year:}}    \\
& Total Referrals     &       3.471&       3.645&       3.565\\
&                     &     (2.497)&     (2.727)&     (2.625)\\
& Unique Specialists  &       2.396&       2.554&       2.482\\
&                     &     (1.318)&     (1.432)&     (1.383)\\
\multicolumn{4}{l}{\emph{Per patient (episode):}}    \\
& 90-day Readmission  &       0.101&       0.083&       0.091\\
&                     &     (0.302)&     (0.276)&     (0.288)\\
& 90-day Mortality    &       0.007&       0.005&       0.006\\
&                     &     (0.082)&     (0.069)&     (0.075)\\
& 90-day Complication &       0.014&       0.006&       0.009\\
&                     &     (0.116)&     (0.076)&     (0.096)\\
& Any Failure         &       0.105&       0.086&       0.095\\
&                     &     (0.307)&     (0.280)&     (0.293)\\
\hline
\end{tabular}

\end{frame}

%%%

\begin{frame}{Referral Concentration}
{Computed per PCP, using six years of data from 2013 to 2018}

\footnotesize

\begin{tabular}{cc}
    \includegraphics[scale=0.3]{results/figures/desc/NetworkSize.png} &
    \includegraphics[scale=0.3]{results/figures/desc/HighestShare.png} \\
    Mean = 7.7 & Mean = 0.38 
\end{tabular}

\bigskip
\bigskip

\small

\emph{Average number of specialists per market: 32}
% Most PCPs refer their patients to less than 1/4 of specialists in their market

\end{frame}

%%%

\begin{frame}{Degree Distribution}
{Computed per PCP, using six years of data from 2013 to 2018}

\footnotesize

\begin{tabular}{cc}
    \includegraphics[scale=0.3]{results/figures/desc/NetworkSize.png} &
        \includegraphics[scale=0.3]{results/figures/desc/LLNetworkSize.png} \\
    Mean = 7.7 & $ \gamma = -3.58$ 
\end{tabular}

\bigskip

\small

\end{frame}

%%%

\begin{frame}{Heterogeneity in quality and cost}
{Interquartile ranges within markets of negative outcomes and episode spending}

\footnotesize

\begin{tabular}{cc}
    \includegraphics[scale=0.3]{results/figures/desc/Failure_IQR.png} &
    \includegraphics[scale=0.3]{results/figures/desc/Payment_IQR.png} \\
    Mean = 0.07 & Mean = \$7,000 
\end{tabular}

\bigskip

\small

\emph{If patients could be reallocated patients from worse to better specialists (75th to 25th percentiles of negative outcomes or episode spending), substantial improvements are possible for those patients.}

\end{frame}



%%%%%%%%%%%%%%%%%%%%%%%%%%%%%%%%%%%%
% MODEL
%%%%%%%%%%%%%%%%%%%%%%%%%%%%%%%%%%%%

\begin{frame}

\vskip24pt

\begin{center}
\LARGE
\color[RGB]{89, 38, 11}
MODEL
\end{center}

\vskip30pt

\small

\emph{PCPs refer a sequence of patients to a set of specialists, and learn about the quality of those specialists from the outcomes for their patients.}

\end{frame}

%%%

\begin{frame}{Actions and outcomes}
\begin{itemize}

\item Each period $t$, the PCP $i$ sends a patient (also $t$) to a specialist $j$ from a fixed choice set $J$
\bit
    \item choice indicators: $D_{ijt} = 0/1$, $j \in J$
\eit
\medskip

\item Specialist treats the patient
\bit
    \item binary outcomes: $Y_{ijt} = 1$ (success) or 0 (failure)
    \item specialist quality: $q_j \equiv \Pr(Y_{ijt} = 1)$
\eit 
\medskip \pause

\item Number of patients sent to specialist $j$ in the past may affect the PCP's utility (e.g., familiarity)
\bit
    \item previous patients: $e_{ijt} \equiv \sum_{s=0}^{t-1} D_{ist}$
\eit 
\medskip

\item Total number of patients currently seen by the specialist may affect the choice probability (e.g., capacity constraints)
% a proxy
\bit
    \item specialist's current patients: $n_{jt} \equiv \sum_k D_{kjt}$
\eit 
% yes it's endogenous

\end{itemize}    
\end{frame}

%%%

\begin{frame}{Learning about quality}
\begin{itemize}

% How we are thinking about this...

\item Bayesian learning, multi-armed bandit framework
\bigskip

\item PCP has beliefs about the quality of each specialist
\bit
    \item beliefs: $q_j \sim \text{Beta}(a,b)$
\eit
% natural distribution for binary outcomes (conjugate prior)
% a and b represent numbers of successes and failures
\bigskip \pause

\item Updating beliefs from experience
\bit
    \item common priors: $(a_0, b_0)$
    % larger sum means stronger priors (more concentrated)
    \item posteriors:
    $a_{ijt} = a_0 + \sum_{s=1}^t Y_{ijs} , \ \ b_{ijt} = b_0 + \sum_{s=1}^t (D_{ijs} - Y_{ijs})$
\eit
\bigskip

\item Mean and variance at start of period $t$:
\bit
    \item mean: $m_{ijt} \equiv \frac{ a_{ij,t-1} }{ a_{ij,t-1} + b_{ij,t-1} }$
    \item variance: $v_{ijt} \equiv \frac{ a_{ij,t-1} b_{ij,t-1} }
    { (a_{ij,t-1} + b_{ij,t-1})^2 (a_{ij,t-1} + b_{ij,t-1} + 1) }
    = \frac{m_{ijt}(1-m_{ijt})}{e_{ijt} + a_0 + b_0 + 1}$
\eit

\end{itemize}
\end{frame}

%%%

\begin{frame}{Referring physician's utility}
\begin{itemize}

\item PCP $i$ utility from sending patient $t$ to specialist $j$ (at date $t$): \\
\smallskip

$U_{ijt} \equiv \alpha Y_{ijt} + f(e_{ijt}) + u(x_{ijt}) + c(n_{jt}) + \xi_j + \epsilon_{ijt}$
\smallskip

\bit
    \item $\alpha$ -- weight on patient outcomes (e.g., altruism)
    \item $f$ -- subjective taste for familiarity
    \item $x$ -- patient-specific factors (e.g., distance)
    \item $c$ -- congestion effect
    \item $\xi$ -- other demand factors (unobserved to econometrician)
    \item $\epsilon$ -- idiosyncratic shock
\eit 
\bigskip \pause

\item Model captures:
\bit
    \item learning about quality
    \item taste for familiarity
    \item capacity constraints
\eit

\end{itemize}
\end{frame}

%%%

\begin{frame}{Referral decisions}

a) Myopic behavior
\[\max_{j \in J} \ \text{E} \left[ U_{ijt} | \dots \right] =
\max_{j \in J} \left\{ \alpha m_{ijt} + f(e_{ijt}) + u(x_{ijt}) + c(n_{jt}) + \xi_j + \epsilon_{ijt} \right\}
\]
\smallskip \pause

b) Forward-looking behavior
\[
V_{it}(\dots) 
= \max_{j \in J} \ \left\{ \text{E} \left[ U_{ijt} | \dots \right]
+ \beta \E V_{i,t+1}(\dots) \right\}
\]

\begin{itemize} \small

\item No dynamics for $x,n,\xi,\epsilon$
\smallskip
\item Gittins index solution (Gittins 1979) for present discounted value of optimal returns from specialist $j$, denote as $g(m_{ijt}, v_{ijt})$

\item Present discounted value of familiarity with $j$, denote as $\overline{\overline{f}}$
\smallskip
\item Result:

\end{itemize}

\vskip-12pt

\[
V_{it}(\dots) 
=\max_{j \in J} \left\{ \alpha g(m_{ijt}, v_{ijt}) + \overline{\overline{f}}(e_{ijt}) + u(x_{ijt}) + c(n_{jt}) + \xi_j + \epsilon_{ijt} \right\}
\]

\end{frame}

%%%

\begin{frame}[t]{Identification of the myopic model}

\vskip-12pt

$$\alpha m_{ijt} + f(e_{ijt}) + u(x_{ijt}) + c(n_{jt}) + \xi_j + \epsilon_{ijt}$$

\begin{itemize}

\item Assume a known distribution for $\epsilon$
\bigskip

\item Choice prob's identify differences in utility btw alternatives
\bit
    \item Effects of exogenous observables $u(x)$ are standard 
    \item Congestion effect $c(n)$ can be identified with instrument for referrals from other PCPs (Richards-Shubik \emph{et al.}~2022).
\eit
\bigskip \pause

\item Specialist fixed effects ($\xi$):
identified by cases where histories and observables are the same for two specialists % (up to normalization; e.g., one fixed effect is set to zero)
\bigskip \pause

\item Altruism parameter ($\alpha$): identified by marginal effect of \emph{observed} success rate, in limit as $e \rightarrow \infty$ (``ID at infinity'')
\[
m_{ijt} 
= \frac{ a_0 + \sum_{s=1}^{t-1} Y_{ijs} }{ a_0 + b_0 + e_{ijt} } 
= \frac{ a_0/e_{ijt} + (\sum_{s=1}^{t-1} Y_{ijs})/e_{ijt} }{ (a_0 + b_0)/e_{ijt} + 1 } 
\]
%Thus cases where a PCP has large past experience with two specialists, but different success rates with them, identify the altruism parameter.

\end{itemize}
\end{frame}

%%%

\begin{frame}[t]{Identification of the myopic model}

\vskip-12pt

$$\alpha m_{ijt} + f(e_{ijt}) + u(x_{ijt}) + c(n_{jt}) + \xi_j + \epsilon_{ijt}$$

\begin{itemize}

\item Strength of priors ($a_0 + b_0$):
how the marginal effect of the success rate changes with experience (i.e., interaction)
\[
\alpha m_{ijt} 
= \alpha \frac{ a_0 }{ (a_0 + b_0) + e_{ijt} } 
+ \alpha \frac{ 1 }{ (a_0 + b_0)/e_{ijt} + 1 } (\sum_{s=1}^{t-1} Y_{ijs})/e_{ijt}
\]
\smallskip \pause

\item Prior ratio ($\frac{a_0}{a_0 + b_0}$): must be calibrated; e.g., from average success probability in a market (then have $a_0, b_0$ separately)
\bigskip \pause

\item Effect of familiarity ($f$): remaining parameter can be identified from marginal effect of experience with a specialist
% Intuitively, we need to fully identify the priors (both the ratio and the strength) in order to recover the effect of familiarity, because we need to know exactly how experience washes away the priors in order to separate that from the effect of familiarity.
\bit
    \item Requires a normalization, e.g.~$f(0) = 0$
    \item Must be bounded as $e \rightarrow \infty$
\eit

\end{itemize}
\end{frame}



%%%%%%%%%%%%%%%%%%%%%%%%%%%%%%%%%%%%
% EMPIRICAL ANALYSIS
%%%%%%%%%%%%%%%%%%%%%%%%%%%%%%%%%%%%

\begin{frame}
\begin{center}
\LARGE
\color[RGB]{89, 38, 11}
EMPIRICAL ANALYSIS
\end{center}
\end{frame}

%%%

\begin{frame}{Event study of specialist's first failure event}
{Effect on referrals per quarter from PCP observing the failure vs.~other PCPs}

\centering
\includegraphics[scale = 0.5]{results/_archive/figures/failure1.png}

\footnotesize

Dep.~var.~mean = 0.096 (unconditional), = 0.22 (conditional).
\iffalse
Sample restricted to PCP/specialist pairs where the specialist ever has a failure (for any PCP).
Balanced panel with 178,903 quarterly observations.
The treated group is the PCP that sent the patient and the control group is all other PCPs that also sent patients to that specialist.
\fi
% When focusing on the first failure in terms of timing, we have 178,903 observations with a mean dependent variable of 0.096 patients per pair per quarter (again, lots of zeros!). When focusing on the second failure, we have 135,188 observations with a mean dependent variable of 0.090. When we stack the first four failure events, we have 466,730 observations with a mean dependent variable of 0.097.

\end{frame}

%%%

\begin{frame}[t]{Logit specification}
\vskip-12pt

\[
    \Pr \left( j  \, | \, \dots \right) = 
    \frac{\exp \left( \pi_1 m_{ijt} + \pi_2 m_{ijt} e_{ijt} + \pi_3 e_{ijt} + \pi_4 x_{ijt} + \xi_j \right)}{\sum_{k \in J_{it}} \exp \left( \pi_1 m_{ikt} + \pi_2 m_{ikt} e_{ikt} + \pi_3 e_{ikt} + \pi_4 x_{ikt} + \xi_k \right)}
\]

\begin{itemize}

\item Multinomial logit with specialist fixed effects
\medskip

\item Variable re-definitions:
\bit
    \item $m_{ijt}$ -- failure rate among patients sent to $j$ over past five years
    \item $e_{ijt}$ -- proportion of patients sent to $j$ over past five years
    \item $x_{ijt}$ -- dist.~from patient to hospital where $j$ primarily operates
\eit
\smallskip

\item Choice set ($J_{it}$): all specialists in market who operate in same year and are w/in 150 miles of patient 
\medskip

\item Identifying variation comes from differences \emph{across} PCPs in outcomes of patients sent to \emph{same} specialists
\medskip

\item Estimated separately in each geographic market (``HRR'')

\end{itemize} 
\end{frame}

%%%

\begin{frame}{Distribution of logit coefficient estimates}
{(276 HRRs; weighted by number of patients in HRR)}
\centering
\footnotesize

With Specialist Fixed Effects
 
\begin{tabular}{lccccccc}
\hline
  & Nat'l & & \multicolumn{5}{c}{Percentile} \\
Variable & Avg. & & 10 & 25 & 50 & 75 & 90 \\
\hline
Past failures (prop.) &  1.071 & &  0.444 &  0.850 &  1.175  &  1.484  &  1.806 \\
          Interaction & -3.688 & & -7.484 & -5.419 & -4.072  & -2.401  & -1.242 \\
Past patients (prop.) &  4.797 & &  3.298 &  4.039 &  4.740  &  5.537  &  6.534 \\
\\
     Distance (miles) & -0.100 & & -0.150 & -0.116 & -0.097  & -0.073  & -0.056 \\
\hline
\end{tabular}

% NOTE: At 0.3 proportion of past patients (mean is 0.30, median is 0.27), mfx of past failures becomes negative.

\bigskip
\medskip

Without Specialist Fixed Effects

\begin{tabular}{lccccccc}
\hline
  & Nat'l & & \multicolumn{5}{c}{Percentile} \\
Variable & Avg. & & 10 & 25 & 50 & 75 & 90 \\
\hline
Past failures (prop.) &  1.529  & &   0.967 &  1.320 &    1.612 &    1.996 &    2.342 \\
          Interaction & -5.929  & &  -9.648 & -7.894 &   -6.405 &   -4.602 &   -2.976 \\
Past patients (prop.) &  6.421  & &   4.920 &  5.667 &    6.447 &    7.218 &    7.966 \\
\\
     Distance (miles) & -0.061  & &  -0.100 & -0.073 &   -0.058 &   -0.041  &  -0.028 \\
\hline
\end{tabular}

\end{frame}

%%%

\begin{frame}{Distribution of marginal effects of past failure rate}
{(276 HRRs; weighted by number of patients in HRR)}
\footnotesize

\begin{tabular}{cc}
With Specialist Fixed Effects &
W/out Specialist Fixed Effects \\
\includegraphics[scale=0.23]{results/_archive/figures/mfx_failures_hist.png} &
\includegraphics[scale=0.23]{results/_archive/figures/mfx_failures_noFE.png} \\
Z-score $< -1.96$: 33/276 (12\%) &
Z-score $< -1.96$: 73/285 (26\%)
\end{tabular}

\end{frame}

%%%

\begin{frame}{National average marginal effects}
{(276 HRRs; weighted by number of patients in HRR)}

\begin{center}
\begin{tabular}{lcc}
     \hline
     & \multicolumn{2}{c}{Model:} \\
     Proportion of: &  With FEs  &  W/out FEs  \\
     \hline
     Past failures  &  -0.0644  &  -0.1202   \\
            &    (0.0057)  &  (0.0061)  \\
     \\
     Past patients  &  0.7028  &  0.8523  \\
            &   (0.0034)  &  (0.0031)  \\
     \hline
\end{tabular}
\end{center}

\bigskip \pause

Interpretation:
\begin{itemize}
\item 2.44 referrals per pair (over six years)
\item 1 failure / 2.44 = 0.41 increase in failure rate
\item 0.38 share of referrals sent to most chosen specialist
\item 0.41 x 0.0644 / 0.38 = 6.9\% relative reduction
\end{itemize}

\end{frame}

%%%

\begin{frame}{Market-level factors and the responsiveness to outcomes}
{Dependent variable: marginal effect of past failure rate, from logit with specialist FEs}
\centering
\scriptsize 

\begin{center}
\begin{tabular}{lccccc}
     \hline
     Explanatory  & (1) & (2) & (3) & (4) & (5) \\
     Variables  \\
     \hline
     Patients  & -0.0037 & 0.0007 & 0.0015 & -0.0026 & -0.0041 \\
     (1000s) & (0.0020) & (0.0029) & (0.0036) & (0.0021) & (0.0020) \\
     \\
     Specialists   & ~ & -0.0158 & ~ & ~ & ~ \\
     (100s) & ~ & (0.0075) & ~ & ~ & ~ \\
     \\
     PCPs   & ~ & ~ & -0.0118 & ~ & ~ \\
     (100s) & ~ & ~ & (0.0068) & ~ & ~ \\
     \\
     Specialist HHI   & ~ & ~ & ~ & 0.0805 & ~ \\
     (0.0--1.0) & ~ & ~ & ~ & (0.0438) & ~ \\
     \\
     Integration   & ~ & ~ & ~ & ~ & -0.0521 \\
     (0.0--1.0) & ~ & ~ & ~ & ~ & (0.0544) \\
     \\
     Observations & 276 & 276 & 276 & 276 & 276 \\
     \hline
     \multicolumn{6}{l}{Each column is a separate regression; weighted by inverse of standard error} \\
     \multicolumn{6}{l}{of dependent variable.}
\end{tabular}
\end{center}

\end{frame}

%%%

\begin{frame}{Responsiveness to outcomes by specialist HHI}
\centering
\small 

\begin{tabular}{cc}
Mean MFX by HHI &
Local Linear Regression \\
\\
\begin{tabular}{ccc}
\hline
HHI  &  MFX  &  \# mkts  \\
\hline
0.00-0.15  &  -0.080  & 122  \\
0.15-0.25  &  -0.059  &  64  \\
0.25-1.00  &  -0.035  &  90  \\
\hline
\vspace{60pt}
\end{tabular}
&
\parbox{0.5\linewidth}{\includegraphics[scale=0.25]{results/_archive/figures/lpoly_HHI.png}}
\end{tabular}

\end{frame}

%%%

\begin{frame}{Summary}
\begin{itemize}

\item Substantial improvements would be possible if referrals can be reallocated \emph{within} geographic areas
\bigskip

\item Structural learning model can quantify frictions and project amount of possible reallocation
\bigskip

\item Using variation in patient outcomes across PCPs referring patients to the same specialists, we estimate a small but significant responsiveness of referrals to outcomes
\bigskip

\item At the market level, responsiveness to outcomes is
\bit
    \item Positively related to the number of physicians
    \item Negatively related to market concentration among specialists
\eit

\end{itemize}
\end{frame}

%%%

\iffalse
\begin{frame}{Persistence of referrals by specialist HHI}
\centering
\small 

\begin{tabular}{cc}
Mean MFX by HHI &
Local Linear Regression \\
\\
\begin{tabular}{lcc}
\hline
HHI  &  MFX  &  \# mkts  \\
\hline
0.0-0.1  &   0.765 &  65  \\
0.1-0.2  &  0.693  &  91  \\
0.2-0.4  &  0.659  &  88  \\
0.4-1.0  &  0.625  &  32  \\
\hline
\vspace{48pt}
\end{tabular}
&
\parbox{0.5\linewidth}{\includegraphics[scale=0.25]{results/_archive/figures/lpoly_inertia_HHI.png}}
\end{tabular}

shop less when you have more choice

\end{frame}
\fi

%%%

\end{document}

%%%

\begin{frame}{title}
\begin{itemize}

\item x

\end{itemize}
\end{frame}
