%%% SETUP

\documentclass[slides, smaller]{beamer}

\setbeameroption{hide notes}
%\setbeameroption{show notes}
%\setbeamertemplate{note page}[plain]
%\setbeameroption{show notes on second screen}

% Theme:

\usetheme{Malmoe}
\usecolortheme[RGB={101, 56, 25}]{structure}

\setbeamertemplate{itemize items}[default]
\setbeamertemplate{itemize subitem}[circle]

\setbeamertemplate{navigation symbols}{}

% Packages:
\usepackage{amssymb,amsmath,amsfonts,amsthm}
\usepackage{graphicx, tikz}
\usepackage{setspace}
\usepackage{xkeyval}

% Commands:
\newcommand{\E}{\text{E}}
\newcommand{\V}{\mathrm{V}}
%\newcommand{\C}{\mathrm{Cov}}
\newcommand{\dd}{\mathrm{d}}
\newcommand{\beq}{\begin{equation}}
\newcommand{\eeq}{\end{equation}}

\newcommand{\bit}{\begin{itemize}}
\newcommand{\eit}{\end{itemize}}

\newcommand{\benu}{\begin{enumerate}}
\newcommand{\eenu}{\end{enumerate}}

\definecolor{themebrown}{RGB}{101, 56, 25}

% Title:

\title[How Efficient is the Market for Physician Referrals? \hspace{0.5cm} \insertframenumber]{How Efficient is the Market for \\ Physician Referrals? \\ \ \\ PRELIMINARY}
\author[McCarthy \and Richards-Shubik]%
{Ian McCarthy\inst{1} \and Seth Richards-Shubik\inst{2}}
\institute[]{
  \inst{1} Emory University and NBER
  \and
  \inst{2} Lehigh University and NBER
}
\date[]{May 2023}

\begin{document}

\frame{\titlepage}

% Thank those who saw me last time.

% This is one of my two new main projects.
% Other one relates to the work I presented here before...

%%% INTRO

\begin{frame}{Physician referrals}
\begin{itemize}

%\item Production network (micro, one service) with substantial informational frictions and heterogeneity in prices and quality
\item Service market where professionals with general skills (e.g., primary care / GPs) direct consumers to professionals with specialized skills
\bit
    \item Substantial heterogeneity  in prices and quality across specialists
    \item Prices do not clear market in the short run
\eit
% Suggests there may be large informational frictions
\bigskip \pause

\item Uncertainty about quality is a defining characteristic of most \\ health care markets:
\smallskip

{\small
%Title of Arrow's seminal paper is Uncertainty and the Welfare Economics of Medical Care
\emph{Uncertainty as to the quality of the product is perhaps more intense here than in any other important commodity.} (Arrow 1963, p.~951)
}
\bigskip \pause

% Our goal is to analyze this uncertainty in physician referrals...
\item We use a structural learning model to measure the effects of uncertainty in physician referrals, and simulate possible reallocations if learning were improved

\end{itemize}
\end{frame}

%%%

\begin{frame}[t]{Background on referrals}
\begin{itemize}

\bigskip

\item Total physician/clinical services $+$ hospital care: \$2 trillion \\
(2019 National Health Expenditure Accounts)
% almost 10% of GDP
\bit
    \item Primary care accounts for only 5\% -- 8\%
    % https://jamanetwork.com/journals/jama/article-abstract/2757218
    % https://jamanetwork.com/journals/jamainternalmedicine/fullarticle/2730351
    % https://jamanetwork.com/journals/jamainternalmedicine/fullarticle/2765245
    
    \parbox{\linewidth}{\footnotesize
    (Reiff et al.~2019, Reid et al.~2019, Martin et al.~2020)
    % one JAMA, two JAMA Internal Medicine
    }
\eit
\bigskip \medskip \pause

\item Referrals have been increasing over time
% Data from: National Ambulatory Medical Care Survey (and hospital version), large nationally representative survey of physician offices, run by NCHS, with 20,000 to 35,000 observations per year
\medskip

\footnotesize

\begin{tabular}{lccccc}
\hline
    & \multicolumn{2}{c}{Rate per Visit} &   &  \multicolumn{2}{c}{Annual Total} \\
    &   1999    &   2009    &   &   1999    &   2009 \\
\hline
Referrals from primary care visits  &   5.8\%   &   9.9\%   &   &   22M &   51M \\
Referrals among specialists         &   2.9\%   &   7.3\%   &   &   11M &   38M \\
Patient self-referrals to specialists & 6.0\%   &   2.8\%   &   &   51M &   31M \\
% self-referral = any new patient visit that was marked as being not referred
\hline
\end{tabular}

\smallskip

\parbox{\linewidth}{\footnotesize
(Barnett et al.~2012)
}

\end{itemize}

\end{frame}

%%% LIT 1
\iffalse 
\begin{frame}[t]{Background on referrals}
\begin{itemize}

\bigskip

\item Referring physicians have large influence on choice of specialist

\parbox{\linewidth}{\footnotesize
(e.g., Chernew, Cooper, Hallock, and Scott Morton, \emph{JHlthEcon} 2021)
}

% This is about choice of where to get MRI (when need it for xxx, referred by orthopedic specialist), a fairly undifferentiated service in terms of quality, among patients in xxx location / plan [data source] with large cost-sharing (deductible).

\bigskip

\end{itemize}

\begin{tikzpicture} 
\node (table) at (0,0)
    {\includegraphics[scale=0.6]{presentations/images/lit_mri.png}};
\only<2>{
\node[rectangle, minimum width=4cm, draw=red] at (-3.1,-1.21)
    {};
}
\end{tikzpicture}

\end{frame}

%%% LIT 2

\begin{frame}[t,noframenumbering]{Background on referrals}
\begin{itemize}

\bigskip

\item Referring physicians have large influence on choice of specialist

\item Referring physicians say patient outcomes and experiences matter

\end{itemize}
\end{frame}

%%% LIT 2.1

\begin{frame}[t,noframenumbering]{Background on referrals}
{(Schneider and Epstein, \emph{NEJM} 1996)}

\centering

\begin{tikzpicture} 
\node (table) at (0,0)
    {\includegraphics[scale=0.55]{presentations/images/lit_rptcard.png}};
\only<2-3>{
\node[rectangle, minimum width=1.5cm, minimum height=6.0cm, draw=red] at (1.85,-0.9)
    {};
}
\only<3>{
\node[rectangle, minimum width=6.6cm, draw=red] at (-0.6,+0.08)
    {};
\node[rectangle, minimum width=6.6cm, draw=red] at (-0.6,-1.26)
    {};
}
\end{tikzpicture}

% Consumer Guide to Coronary Artery Bypass Graft Surgery

\end{frame}

%%% LIT 2.2

\begin{frame}[t,noframenumbering]{Background on referrals}
{(Barnett, Keating, Christakis, O'Malley, and Landon, \emph{JGenIntMed} 2012)}

What is the most or second most important reason, \emph{besides clinical expertise}, for referring patients to [a chosen specialist]? 

\centering

\begin{tikzpicture} 
\node (table) at (0,0)
    {\includegraphics[scale=0.6]{presentations/images/lit_reason.png}};
\only<2>{
\node[rectangle, minimum width=6.8cm, draw=red] at (-1.74,+0.42)
    {};
\node[rectangle, minimum width=6.8cm, draw=red] at (-1.74,-0.69)
    {};
}
\end{tikzpicture}

\end{frame}
\fi
%%% LIT 3

\begin{frame}[t,noframenumbering]{Background on referrals}
\begin{itemize}

\bigskip

\item Referring physicians have large influence on choice of specialist

\item Referring physicians say patient outcomes and experiences matter
\pause
\item Specialist market shares are only weakly associated with quality
\smallskip

\parbox{\linewidth}{\footnotesize
(Kolstad and Chernew, \emph{MedCareRschRev} 2009; 
Dranove, \emph{HandbookHlthEcon} 2011; Skinner, \emph{HandbookHlthEcon} 2011;
Chandra, Finkelstein, Sacarny, and Syverson, \emph{AER}~2016;
Gaynor, Propper, and Seiler, \emph{AER} 2016)
} 

\bit \smallskip
    \item e.g., average marginal rate of substitution between travel distance and mortality risk from heart surgery: \\
    % mixed logits
    \smallskip
    1.8 miles : 1 per 100 risk of death (naive model) \\
    22 miles : 1 per 100 risk of death (account for supply constraints)
    % these are based on noisy point estimates, from a single market
    \smallskip
    
\parbox{\linewidth}{\footnotesize (Richards-Shubik, Roberts, and Donohue, \emph{JHlthEcon} 2022)}
\eit

\end{itemize}
\end{frame}

%%% LIT 4

\begin{frame}[t,noframenumbering]{Background on referrals}
\begin{itemize}

\bigskip

\item Referring physicians have large influence on choice of specialist

\item Referring physicians say patient outcomes and experiences matter

\item Specialist market shares are only weakly associated with quality
%(but \emph{not} public performance ratings, ``report cards'')

\bigskip

\item Greater concentration of a physician's referrals (to fewer specialists) is associated with somewhat lower spending and utilization

% A substantial difference in concentration, betweenness, or density (e.g., 1 SD) is associated with a small difference in spending (5-10\%)

\medskip

\only<1>{

\parbox{\linewidth}{\footnotesize
(Kaur, Perloff, Tompkins, and Bishop, \emph{HlthSvcRsch} 2017; 
Agha, Ericson, Geissler, and Rebitzer, \emph{MgtSci} 2022) 
}

\medskip

\parbox{\linewidth}{\small Similar results using the density of the patient-sharing network or \\
the betweenness centrality of primary care physicians}

\medskip

\parbox{\linewidth}{\footnotesize
(Barnett, Christakis, O'Malley, Onnela, Keating, and Landon, \emph{MedCare} 2012;
Pollack, Weissman, Lemke, Hussey, and Weiner, \emph{JGenIntMed} 2013)
}

}

\iffalse
Magnitudes:
- 1 SD more concentration: 4.5\% lower spending
- 1 SD more concentration: 7.4\% lower spending
- 1 SD more between PCPs (relative to other docs): 6\% lower spending
- low to high tertile of density: 10\% lower spending
\fi 

\iffalse
\smallskip \pause

\item Other factors affecting referral choice:
\bit
    \item peers from medical residency
    \item specialist gender, homophily
    \item vertical integration
\eit

\only<2>{
\smallskip
    
\parbox{\linewidth}{\footnotesize
(Hackl, Hummer, and Prickner, \emph{JHlthEcon} 2015;
Zeltzer, \emph{AEJ:Applied} 2020;
Dossa, Zeltzer, Sutradhar, Simpson, and Baxter, \emph{JAMA Surgery} 2021;
Chen, Orlas, Chang, and Kelleher, \emph{JSurgRsch} 2023;
% Sarsons?
Baker, Bundorf, and Kessler, \emph{JHlthEcon} 2016;
Carlin, Feldman, and Dowd, \emph{HlthEcon} 2016)
}
}
\fi

% NOTE: Focusing on allocation, not decision about whether to refer at all

\end{itemize}
\end{frame}

%%%

\begin{frame}{Our approach}
\begin{itemize}

%\item Physician practice styles / supply-side factors are a large source of inefficient variations in health care
% variations *within* areas

\item We develop a structural learning model for referral decisions and apply it to a specific area of health care with a well defined network of referring physicians and specialists
%(joint replacement)
\smallskip
\iffalse
\bit
%   \item Learning models have been widely applied to study product choice and related problems (Ching, Erdem, and Keane 2013)
%   \item Our model
    \item Model directly relates to a classic theoretical model of variations in treatment choices under uncertainty (Phelps and Mooney 1993)
\eit
\fi

\item Model also accounts for
\bit
    \item Habit persistence (i.e., preference to refer to familiar specialists)
    \item Congestion / capacity constraints at specialists
\eit
\bigskip

\item Will use estimated model to quantify:
\bit
    \item How quickly do referring physicians learn about specialist quality
    \item How much does quality affect referrals, compared to other factors
    \item What is the scope to improve patient allocations, via better learning or information dissemination?
\only<1>{
    \smallskip
%    {\small $\dots$\emph{while accounting for habit persistence and capacity constraints} }
    % for the counterfactuals, it will be important to account for supply (capacity constraints).
}
\eit

\end{itemize}
\end{frame}

%%%

\begin{frame}{Preliminary results}
\begin{itemize}

    \item Descriptively, the wide variation in specialists' outcomes, \emph{within each geographic market}, suggests possible gains:
    % If a patient could be reallocated...
    \bit
        \item[] e.g., average difference btw 75th pctile (bad) and 25 pctile (good) specialist in each market:
        \item[-] rehospitalization rate: 0.046 (+63\% relative)
        % 75th pctile = 1.63 $\times$ 25th pctile \ (4.6\% diff.)
        \item[-] spending per patient: \$7,900 (+46\% relative)
        % 75th pctile = 1.45 $\times$ 25th pctile \ (\$7,900)
        % a bad specialist has 1.63x higher failure rate than a good specialist
        % a high cost specialist has 1.45x more medicare expenses per episode than a low cost specialist
    \eit
    \bigskip
    
    \item Multinomial logits with specialist fixed effects, estimated separately in each geographic market
    \bit
        \item One additional bad outcome reduces referral probability by 3.4\% (relative), on average nationwide
        \item Consistent with the learning model, response to bad outcomes is stronger when PCP has more experience with a specialist
    \eit
    
\end{itemize}
\end{frame}

%%%

\begin{frame}{Related literature}
\begin{itemize}

\item Structural learning models have been widely applied to study product choice and related problems (Ching, Erdem, and Keane 2013)
\bigskip

\item Many studies in economics and marketing have applied Bayesian learning models to prescribing decisions by individual physicians
% Focusing on examples in economics that use forward-looking models:
\bit
    \item New pharmaceutical drugs \\
    \parbox{\linewidth}{\footnotesize (e.g., Ching 2010; Ferreyra and Kosenok 2011)}
    \item Patient-specific drug matches \\
    \parbox{\linewidth}{\footnotesize (e.g., Crawford and Shum 2005; Dickstein 2021)}
    \item Gong (2018) considers a new surgical procedure
\eit
% Typical countefactuals (why need structural models): xxx
% Gittins solution -- Dickstein, Gong (Ferreyra and Kosenok use a similar threshold approach)
\bigskip \pause

\item Other factors affecting referral choice:
% Not in the model I am presenting today, but could potentially be incorporated.
\bit
    \item peers from medical training
    \item distance between offices
    \item specialist gender, homophily % no variation b/c nearly all orthopedic surgeons are men
    \item vertical integration
\eit

%\only<2>{
\smallskip
    
\parbox{\linewidth}{\footnotesize
(Hackl, Hummer, and Prickner, \emph{JHlthEcon} 2015;
Richards-Shubik, Roberts, and Donohue, \emph{JHlthEcon} 2022;
Zeltzer, \emph{AEJ:Applied} 2020;
%Dossa, Zeltzer, Sutradhar, Simpson, and Baxter, \emph{JAMA Surgery} 2021;
%Chen, Orlas, Chang, and Kelleher, \emph{JSurgRsch} 2023;
% Sarsons?
Baker, Bundorf, and Kessler, \emph{JHlthEcon} 2016;
Carlin, Feldman, and Dowd, \emph{HlthEcon} 2016)
}
%}

\end{itemize}
\end{frame}

%%%

\begin{frame}{Plan for the talk}
\begin{itemize}

%\item Outline:
%\bit
    \item Background on application
    \medskip
    \item Bayesian learning model
    \medskip
    \item Identification
    \medskip
    \item Data, description of referral patterns and outcomes
    \medskip
    \item Multinomial logit estimates
%\eit
%\bigskip

\iffalse
\item Results:
\bit
    \item Wide variation in outcomes across specialists within a market:
    % If a patient could be reallocated...
    \bit
        \item[-] 75th pctile rate of negative outcomes = 1.63 $\times$ 25th pctile
        \item[-] 75th pctile spending per patient episode = 1.45 $\times$ 25th pctile
        \item[] (averages across markets)
        %a bad specialist has 1.63x higher failure rate than a good specialist. And a high cost specialist has 1.45x more medicare expenses per episode than a low cost specialist
    \eit
    \item Logit estimates show one additional bad outcome reduces referral probability by 3.4\% (relative), on average across all markets
\eit
\fi

\end{itemize}
\end{frame}



%%%%%%%%%%%%%%%%%%%%%%%%%%%%%%%%%%%%
% BACKGROUND
%%%%%%%%%%%%%%%%%%%%%%%%%%%%%%%%%%%%

\begin{frame}
\begin{center}
\LARGE
\color[RGB]{89, 38, 11}
BACKGROUND
\end{center}
\end{frame}

%%%

\begin{frame}{Application: joint replacement surgery}
\begin{itemize}

\item About 500,000 hip replacements and 1,000,000 knee replacements in the US per year (and rapidly growing)
% e.g., prediction: 780-925k hip and 1.5-2.4m knee in 2030; so a 50-100% increase by then
% https://doi.org/10.3899/jrheum.170990
\bit
    \item Expensive: \$23,000 mean/median (\$12,500 std.~dev.) 
    \item Some risk: 9.3\% rehospitalized, 0.6\% die from complications
    % from our data
\eit
\bigskip

\item Patients are typically referred by their primary care physicians (PCPs) to orthopedic surgeons (specialists) for the procedure
\bigskip \pause

\item Data from Medicare claims, 2008--2018
\bit
    \item Analysis restricted to persons aged 65+
    \item Approx.~50\%/66\% of all hip/knee replacements at the time
\eit
% transaction-level data

\end{itemize}
\end{frame}

%%%
\iffalse
\begin{frame}{Numbers/Rates of Hip and Knee Replacements}{CDC/NCHS estimates}

\only<1>{
\includegraphics[scale=0.3]{presentations/images/db186_fig1.png}
}
% SOURCE: NCHS Data Brief No. 186, February 2015
% https://www.cdc.gov/nchs/products/databriefs/db186.htm

\only<2>{
\includegraphics[scale=0.3]{presentations/images/db210_fig2.png}
}
% SOURCE: NCHS Data Brief No. 210, August 2015
% https://www.cdc.gov/nchs/products/databriefs/db210.htm

\end{frame}
\fi


%%%%%%%%%%%%%%%%%%%%%%%%%%%%%%%%%%%%
% MODEL
%%%%%%%%%%%%%%%%%%%%%%%%%%%%%%%%%%%%

\begin{frame}

\vskip24pt

\begin{center}
\LARGE
\color[RGB]{89, 38, 11}
MODEL
\end{center}

\vskip30pt

\small

\emph{PCPs refer a sequence of patients to a set of specialists, and learn about the quality of those specialists from the outcomes of their patients.}

\end{frame}

%%%

% Read Ma's referral model in RAND
% Think about other classic theory ref's

\begin{frame}{Actions and outcomes}
\begin{itemize}

\item Each period $t$, the PCP $i$ sends a patient (also $t$) to a specialist $j$ from a fixed choice set $J$
\bit
    \item choice indicators: $D_{ijt} = 0/1$, $j \in J$
\eit
\medskip

\item Specialist treats the patient
\bit
    \item binary outcomes: $Y_{ijt} = 1$ (success) or 0 (failure)
    \item specialist quality: $q_j \equiv \Pr(Y_{ijt} = 1)$
\eit 
\medskip \pause

\item Number of patients sent to specialist $j$ in the past may affect the PCP's utility (e.g., familiarity)
\bit
    \item previous patients: $e_{ijt} \equiv \sum_{s=0}^{t-1} D_{ist}$
\eit 
\medskip

\item Total number of patients currently seen by the specialist may affect the choice probability (e.g., capacity constraints)
% a proxy
\bit
    \item specialist's current patients: $n_{jt} \equiv \sum_k D_{kjt}$
\eit 
% yes it's endogenous

\end{itemize}    
\end{frame}

%%%

\begin{frame}{Learning about quality}
\begin{itemize}

\item Bayesian learning, multi-armed bandit framework
\bit
    \item Beta-binomial model, as proposed in the literature to study physician learning and treatment variations (Phelps and Mooney 1992)
    \item Gittins index (Gittins 1979, Gittins and Jones 1979) provides highly tractable solution for forward looking model
\eit
\medskip \pause

\item PCP beliefs about specialist quality: $q_j \sim \text{Beta}(a,b)$
\bit
    \item common, homogeneous priors: $(a_0, b_0)$
    % larger sum means stronger priors (more concentrated)
    \item updating beliefs from patient outcomes: \\
    \vskip6pt
    $a_{ijt} = a_0 + \sum_{s=1}^t Y_{ijs} , \ \ b_{ijt} = b_0 + \sum_{s=1}^t (D_{ijs} - Y_{ijs})$
\eit
% natural distribution for binary outcomes (conjugate prior)
% a and b represent numbers of successes and failures
\medskip \pause

\item Relationship to the existing literature
\bit
    \item Normal-normal models of prescribing (e.g., Crawford and Shum 2005, Ching 2010, Ferreyra and Kosenok 2011) -- choice sets binary to moderate size ($\sim$5-7), observe seq.~of choices but not outcomes
    \smallskip
    \item Beta-binomial models with Gittins index solutions:
    \item[] Dickstein (2021), antidepressants -- larger choice set (19), nested structure with correlated beliefs, outcomes not observed;
    % Uses binary outcomes for tractability: In modeling outcomes as discrete events, I depart from previous empirical learning models, including Erdem and Keane (1996), Crawford and Shum (2005), and Ching (2010). I do so primarily to simplify the estimation procedure. (p. 12)
    % Outcomes not observed:  In the empirical application, the number of success and failures are not observed; I integrate over the discrete number of possible realizations when computing the likelihood. (p. 12)
    \item[] Gong (2018), brain surgery -- 3 options, learning by doing
\eit

\end{itemize}
\end{frame}

%%%

\begin{frame}{Referring physician's utility}
\begin{itemize}

\item PCP $i$ utility from sending patient $t$ to specialist $j$ (at date $t$): \\
\smallskip

$U_{ijt} \equiv \alpha Y_{ijt} + f(e_{ijt}) + u(x_{ijt}) + c(n_{jt}, z_j) + \xi_j + \epsilon_{ijt}$
\smallskip

\bit
    \item $\alpha$ -- weight on patient outcomes (e.g., altruism)
    \item $f$ -- subjective taste for familiarity
    \item $x$ -- patient-specific factors (e.g., distance)
    \item $c$ -- congestion effect (indiv.~specialists have limited supply)
    % (as in Richards-Shubik \emph{et al.}~2022)
    \item $z$ -- factors affecting capacity (e.g., max operating days per month)
    \item $\xi$ -- other demand factors (unobserved to econometrician)
    \item $\epsilon$ -- idiosyncratic shock
\eit 
\bigskip \pause

\item Model captures key features related to the potential to improve market allocations:
\bit
    \item learning about quality
    \item taste for familiarity
    \item capacity constraints
\eit

\end{itemize}
\end{frame}

%%%

\begin{frame}[t]{Referral decisions}

a) Myopic behavior
\[\max_{j \in J} \ \text{E} \left[ U_{ijt} | \dots \right] =
\max_{j \in J} \left\{ \alpha m_{ijt} + f(e_{ijt}) + u(x_{ijt}) + c(n_{jt}, z_j) + \xi_j + \epsilon_{ijt} \right\}
\]
%\pause

\begin{itemize}

\item Use mean of beliefs at start of period $t$:

{\footnotesize \color{gray} (b/c $\E[Y_{ijt} | (D_{ijs}, Y_{ijs})_{s=1}^{t-1}] = \E[q_j | (D_{ijs}, Y_{ijs})_{s=1}^{t-1}]$) }

\begin{align*}
    m_{ijt} \equiv \E[q_j | (D_{ijs}, Y_{ijs})_{s=1}^{t-1}] & = \frac{ a_{ij,t-1} }{ a_{ij,t-1} + b_{ij,t-1} } \\[6pt]
    & = \frac{ a_0 +  \sum_{s=1}^{t-1} Y_{ijs} }{ a_0 + b_0 + \sum_{s=1}^{t-1} D_{ijs} }
%    & = \frac{ a_0 +  y_{ijt}}{ a_0 + b_0 + e_{ijt} }
\end{align*}

% where $y_{ijt} \equiv \sum_{s=1}^{t-1} Y_{ijs}$

\end{itemize}

\iffalse
\bit
    \item variance: $v_{ijt} \equiv \frac{ a_{ij,t-1} b_{ij,t-1} }
    { (a_{ij,t-1} + b_{ij,t-1})^2 (a_{ij,t-1} + b_{ij,t-1} + 1) }
    = \frac{m_{ijt}(1-m_{ijt})}{e_{ijt} + a_0 + b_0 + 1}$
\eit
\fi 

\end{frame}

%%%

\begin{frame}[t]{Referral decisions}

b) Forward-looking behavior
\[
V_{it}(\dots) 
= \max_{j \in J} \ \left\{ \text{E} \left[ U_{ijt} | \dots \right]
+ \beta \E V_{i,t+1}(\dots) \right\}
\]
\vskip-3pt \pause

\begin{itemize}

\item No dynamics for $x,n,\xi,\epsilon$
% n - because individual PCP is small portion of surgeon's total patient volume
\medskip \pause

\item Gittins index solution (Gittins 1979) for present discounted value of optimal returns from specialist $j$, denote as $g(m_{ijt}, v_{ijt})$
\bit
    \item Assumptions: \\ 
    1) one option is chosen at a time \\
    2) the unchosen options do not affect the current outcome \\
    % For 1 and 2: patient treated by one surgeon, and other surgeons do not affect the outcome.
    3) the expected returns do not change for the unchosen options \\
    4) beliefs are independent across options \\
    ($q_j$ fixed, no spillovers among surgeons $\implies$ 3 \& 4)
    % Dickstein (2021) p. 15: "Gittins and Jones (1979) [WRONG CITE?] prove that the forward induction rule solves the sequence problem in (3.5) if the following conditions hold: (1) the decision-maker selects one option at t; (2) the options not selected do not contribute to the individual’s outcome; (3) the options not selected will produce the same average outcome in later periods as they would if chosen in the initial period; and, (4) the options are independent. Independence in this context implies that the probability of a successful outcome from treatment j is independent of the probability of success on treatment k."
    % Gong (2018) p. 20: "induction. Gittins (1979) proposes an index policy for the classic MAB model. He calculates for each arm an index that only depends on the arm’s current state and calibrates the value of pulling it until some optimal stopping time. The Gittins index policy simply selects the arm with the highest index in each period. Gittins then shows the index policy is optimal in the standard MAB framework, which requires four assumptions: (a) exactly one arm is chosen (or active) in each period; (b) the unchosen arms do not generate rewards; (c) states of the unchosen arms remain frozen, generating the same average rewards in later periods; and (d) the arms are independent."
    \pause 
    \item Brezzi and Lai (2002) provide a closed-form approximation to $g$
\eit
\medskip

\item Result:
\vskip-12pt

\[
V_{it}(\dots) 
=\max_{j \in J} \left\{ \alpha g(m_{ijt}, v_{ijt}) + \overline{\overline{f}}(e_{ijt}) + u(x_{ijt}) + c(n_{jt}, z_j) + \xi_j + \epsilon_{ijt} \right\}
\]

\vskip-6pt

where $\overline{\overline{f}}$ is the present discounted value of familiarity with $j$
% and v is the variance of the beliefs, which has a simple formula

\end{itemize}

\end{frame}

%%%

\begin{frame}[t]{Identification of the myopic model} \label{utility specification}

% NOTE: This formal analysis is new.  Prior papers (Dickstein 2021, Gong 2018) have thoughtful discussions but do not provide formal analysis.

\vskip-12pt

$$\alpha m_{ijt} + f(e_{ijt}) + u(x_{ijt}) + c(n_{jt}, z_j) + \xi_j + \epsilon_{ijt}$$

\begin{itemize}

\item Assume a known distribution for $\epsilon$
\bigskip

\item Choice prob's identify differences in utility btw alternatives
\bit
    \item Specialist fixed effects ($\xi$):
    identified by cases where histories and observables are the same for two specialists 
    % (up to normalization; e.g., one fixed effect is set to zero)
    \item Effects of patient characteristics $u(x)$: assume patients arrive exogenously
\eit
\bigskip \pause

\item Congestion effect $c(n, z)$ can be identified with instrument for referrals from other PCPs (Richards-Shubik \emph{et al.}~2022) 
\ \hyperlink{congestion}{\beamerbutton{more}}

\end{itemize}
\end{frame}

%%%

\begin{frame}[t]{Identification of the myopic model}

\vskip-12pt

$$\alpha m_{ijt} + f(e_{ijt}) + u(x_{ijt}) + c(n_{jt}, z_j) + \xi_j + \epsilon_{ijt}$$

\begin{itemize}

\item Altruism parameter ($\alpha$): identified by marginal effect of \emph{observed} success rate, in limit as $e \rightarrow \infty$ (``ID at infinity'')
\only<1-2>{ %\only<1-4>{
\[
\alpha m_{ijt} 
= \alpha \frac{ a_0 +  \sum_{s=1}^{t-1} Y_{ijs} }{ a_0 + b_0 + \sum_{s=1}^{t-1} D_{ijs} }
= \alpha \frac{ a_0/e_{ijt} + \bar y_{ijt} }{ (a_0 + b_0)/e_{ijt} + 1 } 
\]
where $\bar y_{ijt} \equiv \sum_{s=1}^{t-1} Y_{ijs} / \sum_{s=1}^{t-1} D_{ijs}$
(and $e_{ijt} \equiv \sum_{s=1}^{t-1} D_{ijs}$)
\vskip3pt
}
% Thus cases where a PCP has large past experience with two specialists, but different success rates with them, identify the altruism parameter.
\medskip \pause

\only<2>{
\item[*] {\small Both Dickstein (2021) and Gong (2018) omit $\alpha$, which fixes the MRS between patient outcomes and other factors (incl.~$\epsilon$) at an assumed value}
% NOTE: This fixes the weight on patient outcomes to 1, with an assumed scale of the error term, so this is not an innocuous normalization.
}
\pause

\item Strength of priors ($a_0 + b_0$):
how the marginal effect of the success rate changes with experience (i.e., interaction)
\hyperlink{identification details}{\beamerbutton{algebra}}

\only<1>{ %\only<3-4>{
\[
\alpha m_{ijt} 
= \alpha \frac{ a_0 }{ (a_0 + b_0) + e_{ijt} } 
+ \alpha \frac{ e_{ijt} }{ (a_0 + b_0) + e_{ijt} } \times \bar y_{ijt}
\]
\vskip-8pt
}
\medskip

\only<3-4>{ %\only<4-5>{
\item Prior mean ($\frac{a_0}{a_0 + b_0}$): estimate outside the model, using average success probability in each market (then have $a_0, b_0$ separately)
\medskip
}

\only<4>{ %\only<5>{
\item Effect of familiarity ($f$): remaining parameter can be identified from marginal effect of experience with a specialist
% Intuitively, we need to fully identify the priors (both the ratio and the strength) in order to recover the effect of familiarity, because we need to know exactly how experience washes away the priors in order to separate that from the effect of familiarity.
\bit
    \item Requires a normalization, e.g.~$f(0) = 0$
    \item Must be bounded as $e \rightarrow \infty$
\eit
}

\label{identification}

\end{itemize}
\end{frame}

%%%

\begin{frame}[t]{Conjecture -- extending to the forward-looking model}

\vskip-8pt

$$\alpha g(m_{ijt}, v_{ijt}) + \overline{\overline{f}}(e_{ijt}) + u(x_{ijt}) + c(n_{jt}, z_j) + \xi_j + \epsilon_{ijt}$$

\smallskip %\pause

\begin{itemize}

\item $u(x_{ijt})$, $c(n_{jt}, z_j)$, $\xi_j$ -- same as for myopic model (no dynamics)
\bigskip

\item Gittins index is a known transformation of $m$ and $v$
\bit
    \item[] where:
\[
    v \equiv \frac{ a \, b }
    { (a + b)^2 (a + b + 1) }
    = \frac{m \, (1-m)}{a_0 + b_0 + e + 1}
\]
    \item[] and:
\[
    m = \frac{ a_0 + e \times \bar y}
    { a_0 + b_0 + e  }
\]
\iffalse
    \item[] hence:
\[
    \alpha g(m_{ijt}, v_{ijt}) + \overline{\overline{f}}(e_{ijt}) = \frac{m_{ijt}(1-m_{ijt})}{e_{ijt} + a_0 + b_0 + 1}
\]
\fi
\eit

\item[] hence $\alpha g(m,v)$ is a function of the same unknown parameters ($\alpha, a_0, b_0$) and observables ($e, \bar y$) as the analogous term ($\alpha m$) in the myopic model

\end{itemize}
\end{frame}

%%%


\begin{frame}{Counterfactuals}
\begin{itemize}

\item Faster learning -- reduce strength of priors:

e.g., $ (a_0, b_0) \rightarrow  (a_0 / 2, b_0 / 2) $

\bigskip

\item Complete information -- replace $g(m_{ijt}, v_{ijt})$ with $g(q_j, 0)$

(measures losses due to uncertainty about quality)

\bigskip

\item Forced (or incentivized) experimentation -- assign some fraction of patients to exogenously selected specialists

\bigskip \pause

\item Use estimated model to compute aggregate changes in health outcomes and expenditures
\bit
    \item accounting for congestion / capacity at each specialist
    \item and for habit persistence, taste for familiarity
\eit


\end{itemize}
\end{frame}


%%%%%%%%%%%%%%%%%%%%%%%%%%%%%%%%%%%%
% DATA
%%%%%%%%%%%%%%%%%%%%%%%%%%%%%%%%%%%%

\begin{frame}
\begin{center}
\LARGE
\color[RGB]{89, 38, 11}
DATA
\end{center}
\end{frame}

%%%

\begin{frame}{Data: FFS Medicare claims, 2008--2018}
\begin{itemize}

% transaction-level data on millions of purchases...
\item All fee-for-service Medicare claims for major joint replacements from 2008 to 2018: 4.5 million surgeries
\bigskip

\item Referring physician (PCP) inferred from regular office visits over prior year (2 or more)
\bit
    \item 68\% of surgeries are matched to a PCP
    \item Also restrict to PCPs and specialists with sufficient patient volume 
    % (PCP: 20+ total referrals, specialist: 20+ surgeries per year)  
    % PCP: at least 20 total referrals over our study period and at least three consecutive years with observed referrals
    % specialist: perform at least 20 surgeries per year (captures xxx\% of all surgeries)
    \item 2,014,300 surgeries, 10,500 surgeons, 51,700 PCPs in final sample
\eit
\bigskip \pause

\item Note -- the claims include a ``referring physician'' field, but it is highly unreliable: it is omitted on 34\% of the joint replacement claims, and it lists the \emph{same person} as the operating physician on 64\% of the claims where it is populated.
\bigskip

%\item Comparison to referrals for heart surgery (Richards-Shubik \emph{et al.}~2022): cardiologists refer to surgeons, can be matched based on a pre-surgery imaging procedure (xxx\% match)

\end{itemize}
\end{frame}

%%%

\begin{frame}{Key summary statistics}

\footnotesize
\centering

\begin{tabular}{llrrr}
\hline
&    & (Baseline)     & (Estimation) \\
&    & 2008-2012     & 2013-2018 & Overall\\
\hline
\multicolumn{4}{l}{\emph{Per patient (episode):}}    \\
& 90-day Readmission  &       0.104&       0.085&       0.093\\
& 90-day Complication &       0.014&       0.013&       0.013\\
& 90-day Mortality    &       0.007&       0.005&       0.006\\
& Any Failure         &       0.108&       0.088&       0.096\\[4pt]
& Episode spending   &         23,240&       22,557&       22,839\\
&                     &     (12,717)&     (12,284)&     (12,469)\\[6pt]
\multicolumn{4}{l}{\emph{Per PCP, per year:}}    \\
& Total Referrals     &       3.809&       4.398&       4.134\\
&                    &     (2.940)&     (3.478)&     (3.261)\\
& Unique Specialists  &       2.666&       3.109&       2.910\\
&                    &     (1.580)&     (1.831)&     (1.737)\\[6pt]
\multicolumn{4}{l}{\emph{Per PCP-specialist pair, past five years:}}    \\
& Total Referrals     & - - -    &        4.445 & - - -\\
&                     &     &     (10.453)\\
& Failure Rate  & - - -    &       0.107 & - - -\\
&                     &     &     (0.209)\\
% Last panel does not count zeros...
\hline
\end{tabular}

\end{frame}

%%%

\begin{frame}{Referral Concentration}
{Computed per PCP, using six years of data from 2013 to 2018}

\footnotesize

\begin{tabular}{cc}
    \includegraphics[scale=0.25]{results/figures/desc/NetworkSize_1_1_0.png} &
    \includegraphics[scale=0.25]{results/figures/desc/HighestShare_1_1_0.png} \\
    Mean = 9.6 & Mean = 0.33
\end{tabular}

\bigskip
\bigskip

\small

\emph{Average number of specialists per market: 49}
% Most PCPs refer their patients to less than 1/4 of specialists in their market

\end{frame}

\begin{frame}{Heterogeneity in quality and cost}
{Interquartile ranges within markets of negative outcomes and episode spending}

\footnotesize

\begin{tabular}{cc}
    \includegraphics[scale=0.25]{results/figures/desc/Failure_IQR_1_1_0.png} &
    \includegraphics[scale=0.25]{results/figures/desc/Payment_IQR_1_1_0.png} \\
    Mean = 0.046 & Mean = \$7,932
\end{tabular}

\bigskip

\small

\emph{If patients could be reallocated patients from worse to better specialists \\
(75th to 25th percentiles of negative outcomes or episode spending), substantial improvements are possible for those patients.}

\end{frame}



%%%%%%%%%%%%%%%%%%%%%%%%%%%%%%%%%%%%
% EMPIRICAL ANALYSIS
%%%%%%%%%%%%%%%%%%%%%%%%%%%%%%%%%%%%

\begin{frame}

\vskip48pt

\begin{center}
\LARGE
\color[RGB]{89, 38, 11}
EMPIRICAL ANALYSIS
\end{center}

\vskip18pt

\small

\begin{enumerate}

\item \emph{Design-based evidence of learning from PCP's own patients}

\item \emph{Multinomial logits estimated separately by market}
\bit
    \item[-] \emph{include specialist fixed effects}
    \item[-] \emph{use variation across PCPs referring to same specialist}
\eit


\end{enumerate}

\end{frame}

%%%

\begin{frame}[t]{Event study of specialist's first failure event}
{Effect on referrals per quarter from PCP observing the failure vs.~other PCPs}
\begin{itemize}

\item Within each specialist, compute the difference in referral rates between PCPs who did not observe first failure and PCP who did:
\bit
    \item Define: $I_j$ = PCPs who ever refer to specialist $j$; \\
    $F_j$ = PCP who observes first failure (at time $t_j^1$); 
    and $C_j = I_j \setminus F_j$ 
    \item Compute referral rates from $C_j$ and $F_j$ to $j$ in each quarter: $\bar d^C_{jt}$, $\bar d^F_{jt}$
    \item Take the difference in each quarter, up to the time when the specialist has a second failure: $\bar d^F_{jt} - \bar d^C_{jt}$, \ $t = t_j^0 \dots (t_j^2 - 1$)
    \item Compute the average difference across specialists, using quarters relative to first failure ($q = t - t^1_j$)
\eit
\medskip \pause

\item Corresponds to the following panel model:
\vskip-6pt
$$D_{ijt} = \alpha_{jt} I_{ij} + \gamma F_{ij} + \sum_{q = -9}^{Q_j} \beta_q F_{ij} 1(t - t_j^1 = q) + U_{ijt}$$

\small 

where $I_{ij} = 1(i \in I_j)$ and $F_{ij} = 1(i \in F_j)$
\medskip

($D_{ijt}$ here reflects both the arrival of a patient to PCP $i$ and the decision to refer that patient to specialist $j$)

\end{itemize}
\end{frame}

%%%

\begin{frame}{Event study of specialist's first failure event}
{Effect on referrals per quarter from PCP observing the failure vs.~other PCPs}

\centering
\includegraphics[scale = 0.5]{results/figures/rf/EventStudy_Group1_1_1_0.png}

\footnotesize

Dependent var.~mean = 0.092 (all), 0.194 (treated group before failure)
\iffalse
Sample restricted to PCP/specialist pairs where the specialist ever has a failure (for any PCP).
Balanced panel with xxx quarterly observations.
The treated group is the PCP that sent the patient with the failure and the control group is all other PCPs that also sent patients to that specialist.
\fi
% When focusing on the first failure in terms of timing, we have 178,903 observations with a mean dependent variable of 0.096 patients per pair per quarter (again, lots of zeros!). When focusing on the second failure, we have 135,188 observations with a mean dependent variable of 0.090. When we stack the first four failure events, we have 466,730 observations with a mean dependent variable of 0.097.

\end{frame}

%%%

\begin{frame}[t]{Logit specification}
\vskip-12pt

\[
    \Pr \left( j  \, | \, \dots \right) = 
    \frac{\exp \left( \pi_1 \bar y_{ijt} + \pi_2 \bar y_{ijt} \bar e_{ijt} + \pi_3 \bar e_{ijt} + \pi_4 x_{ijt} + \xi_j \right)}{\sum_{k \in J_{it}} \exp \left( \pi_1 \bar y_{ikt} + \pi_2 \bar y_{ikt} \bar e_{ikt} + \pi_3 \bar e_{ikt} + \pi_4 x_{ikt} + \xi_k \right)}
\]

\begin{itemize}

\item Multinomial logit with specialist fixed effects
\medskip

\item Variable (re-)definitions:
\bit
    \item $\bar y_{ijt}$ -- failure rate among patients sent to $j$ over past five years
    \item $\bar e_{ijt}$ -- proportion of patients sent to $j$ over past five years
    \item $x_{ijt}$ -- dist.~from patient to hospital where $j$ primarily operates
\eit
\smallskip \pause

\item Estimated separately in each geographic market (``HRR'')
\medskip

\item Choice set ($J_{it}$): all specialists in market who operate in same year and are within 75 miles of patient 
\medskip

\item Identifying variation comes from differences in patient histories \emph{across} PCPs in outcomes of patients sent to \emph{same} specialists

\end{itemize} 
\end{frame}

%%% COEFS WITH FEs:

\begin{frame}{Distribution of logit coefficient estimates}
{(297 HRRs; weighted by number of patients in HRR)}
%%%FIX: Actually 299 HRRs here, but 2 have 0 SE of MFX, dropped later.
\centering
\footnotesize

\begin{tabular}{lccccccc}
\hline
  & Nat'l & & \multicolumn{5}{c}{Percentile} \\
Variable & Avg. & & 10 & 25 & 50 & 75 & 90 \\
\hline
Past failures (prop.) &  1.002 & &  0.515 &  0.760 &  1.012 &  1.302 &  1.482 \\
          Interaction & -4.472 & & -7.020 & -6.056 & -4.588 & -3.018 & -1.763 \\
Past patients (prop.) &  5.752 & &  3.998 &  4.750 &  5.642 &  6.889 &  7.672 \\
\\
     Distance (miles) & -0.100 & & -0.141 & -0.121 & -0.097 & -0.078 & -0.064 \\
\hline
\end{tabular}

\bigskip \pause

\begin{itemize}

\small 

\item With national average coef's, the marginal effect of failure rate is negative if at least 22.4\% of past patients have been sent to a specialist
\medskip

\item Negative interaction term is consistent with the learning model
\vskip2pt

(e.g., effect of failure rate in myopic model):
% Model implies marginal effect of failure rate is stronger when PCP has more experience with a specialist:
\[
\alpha m_{ijt} 
= \alpha \frac{ a_0 }{ (a_0 + b_0) + e_{ijt} } 
+ \alpha \frac{ e_{ijt} }{ (a_0 + b_0) + e_{ijt} } \times \bar y_{ijt}
\]

\end{itemize}

\end{frame}

%%%

\begin{frame}{Distribution of marginal effects of past failure rate}
{(297 HRRs; weighted by number of patients in HRR)}

\centering

\includegraphics[scale=0.2]{results/_archive/figures/mfx_failures_hist.png}

\footnotesize

z-score $< -1.96$: 63/297 (21\%)

\end{frame}

%%%

\begin{frame}{National average marginal effects}
%{(299 HRRs)}

\begin{center}
\begin{tabular}{lcc}
     \hline
     Variable       &   Patient-        &  \\
     (as proportions) &  Weighted  &  Unweighted  \\
     \hline
     Past failures  &  -0.0502  &  -0.0396   \\
            &    (0.0027)  &  (0.0038)  \\
     \\
     Past patients  &  0.7481  &  0.7047  \\
            &   (0.0026)  &  (0.0030)  \\
     \hline
\end{tabular}
\end{center}

\bigskip \pause

Interpretation:
\begin{itemize}
\item 4.445 referrals per pair (over six years)
\item 1 failure / 4.445 = 0.225 increase in failure rate
    % (that's almost the same as 1 SD = 0.209)
\item 0.33 average share of referrals sent to most chosen specialist
\item 0.225 x 0.0502 / 0.33 = 3.4\% relative reduction
\end{itemize}

% QUES: Could we get this using p(1-p) b, using p=0.33 and b = b1 + b2*0.33?
% Semi-elasticity interpretation?

\end{frame}

%%%

\begin{frame}{Summary} \label{summary}
\begin{itemize}

\item Substantial improvements would be possible if referrals can be reallocated \emph{within} geographic areas
\bigskip

\item Structural learning model can quantify losses due to uncertainty and project gains from possible reallocation
\bigskip

\item Using variation in patient outcomes across PCPs referring patients to the same specialists, we estimate a small but significant responsiveness of referrals to outcomes
\bigskip

\item At the market level, responsiveness to outcomes is negatively related to market concentration among specialists
\hyperlink{market-level factors}{\beamerbutton{appendix}}

\end{itemize}
\end{frame}

%%%%%%%%%%%%%%%%%%%%%%%%%%%%%%%%%%%%
% APPENDIX SLIDES
%%%%%%%%%%%%%%%%%%%%%%%%%%%%%%%%%%%%

\begin{frame}[noframenumbering]
\begin{center}
\LARGE
\color[RGB]{89, 38, 11}
APPENDIX SLIDES
\end{center}
\end{frame}

%%%

\begin{frame}{Congestion effect}
{(Richards-Shubik, Roberts, and Donohue 2022)} \label{congestion}
\begin{itemize}

\item Based on Bayer and Timmins (2007) spatial equilibrium model:
spillovers among consumers who choose the same ``location''

\item Approach is similar to BLP demand estimation,
but congestion effect takes the place of a price
\bigskip

\item Estimate multinomial logit with specialist-time fixed effects ($\delta_{jt}$) 
\[
\alpha m_{ijt} + f(e_{ijt}) + u(x_{ijt}) + \underbrace{ c(n_{jt}, z_j) + \xi_j }_{\delta_{jt}} + \epsilon_{ijt}
\]

\item Recover congestion effect using regression of estimated fixed effects:
\[
\hat \delta_{jt} = c(n_{jt}, z_j) + \xi_j + \eta_{jt}
\]
\vskip-4pt
\bit
    \item instrument for $n_{jt}$ is distances from \emph{other} patients to specialist $j$
    \item e.g., 2SLS estimation if $c$ is linear
\eit
\medskip

\hyperlink{utility specification}{\beamerbutton{back}}

\end{itemize}
\end{frame}

%%%

\begin{frame}[noframenumbering, t]{Identification of the myopic model}
\label{identification details}

\vskip-12pt

$$\alpha m_{ijt} + f(e_{ijt}) + u(x_{ijt}) + c(n_{jt}, z_j) + \xi_j + \epsilon_{ijt}$$

\begin{itemize}

\item Altruism parameter ($\alpha$): identified by marginal effect of \emph{observed} success rate, in limit as $e \rightarrow \infty$ (``ID at infinity'')
\[
\alpha m_{ijt} 
= \alpha \frac{ a_0 +  \sum_{s=1}^{t-1} Y_{ijs} }{ a_0 + b_0 + \sum_{s=1}^{t-1} D_{ijs} }
= \alpha \frac{ a_0/e_{ijt} + \bar y_{ijt} }{ (a_0 + b_0)/e_{ijt} + 1 } 
\]
where $\bar y_{ijt} \equiv \sum_{s=1}^{t-1} Y_{ijs} / \sum_{s=1}^{t-1} D_{ijs}$
(and $e_{ijt} \equiv \sum_{s=1}^{t-1} D_{ijs}$)
\vskip3pt
% Thus cases where a PCP has large past experience with two specialists, but different success rates with them, identify the altruism parameter.
\medskip

\item Strength of priors ($a_0 + b_0$):
how the marginal effect of the success rate changes with experience (i.e., interaction)
\[
\alpha m_{ijt} 
= \alpha \frac{ a_0 }{ (a_0 + b_0) + e_{ijt} } 
+ \alpha \frac{ e_{ijt} }{ (a_0 + b_0) + e_{ijt} } \times \bar y_{ijt}
\]
\vskip-8pt
\medskip

% IF TIME: Make appendix slide with this algebra
% \beamerbutton{algebra}

\item Prior mean ($\frac{a_0}{a_0 + b_0}$): estimate outside the model, using average success probability in each market (then have $a_0, b_0$ separately)
\medskip

\hyperlink{identification}{\beamerbutton{back}}

\end{itemize}
\end{frame}

%%%

\begin{frame}{Market-level factors and the responsiveness to outcomes}
{Dependent variable: marginal effect of past failure rate, from logit with specialist FEs}
\label{market-level factors}

\centering
\footnotesize

\begin{center}
\begin{tabular}{lcccccc}
     \hline
     Explanatory  & (1) & (2) & (3) & (4) & (5) & (6) \\
     Variables  \\
     \hline
        Patients & -0.0021 & -0.0025 & -0.0024 & -0.0016 & -0.0019 & -0.0023 \\ 
        (1000s) & (0.0009) & (0.0006) & (0.0006) & (0.0006) & (0.0009) & (0.0005) \\[6pt] 
        Specialists & -0.0006 & ~ & ~ & ~ & 0.0009 & ~ \\ 
        (100s) & (0.0023) & ~ & ~ & ~ & (0.0023) & ~ \\[6pt] 
        Spec's/Pat's & ~ & -0.0645 & ~ & ~ & ~ & ~ \\ 
        (ratio) & ~ & (0.1247) & ~ & ~ & ~ & ~ \\[6pt] 
        Spec's/PCPs & ~ & ~ & -0.0071 & ~ & ~ & ~ \\ 
        (ratio) & ~ & ~ & (0.0225) & ~ & ~ & ~ \\[6pt] 
        Specialist HHI & ~ & ~ & ~ & 0.0792 & 0.0821 & ~ \\ 
        (0.0--1.0) & ~ & ~ & ~ & (0.0293) & (0.0302) & ~ \\[6pt] 
        Integration & ~ & ~ & ~ & ~ & ~ & 0.0082 \\ 
        (0.0--1.0) & ~ & ~ & ~ & ~ & ~ & (0.0277) \\[6pt] 
        Observations & 297 & 297 & 297 & 297 & 297 & 297 \\ 
     \hline
     \multicolumn{7}{l}{Each column is a separate regression; weighted by inverse standard error of dep.~variable.} \\
\end{tabular}
\end{center}

\end{frame}

%%%

\begin{frame}{Responsiveness to outcomes by specialist HHI}

\centering
\small 

\begin{tabular}{cc}
Mean MFX by HHI &
Local Linear Regression \\
\\
\begin{tabular}{ccc}
\hline
HHI  &  MFX  &  \# mkts  \\
\hline
0.00-0.15  &  -0.057  & 137  \\
0.15-0.25  &  -0.045  &  81  \\
0.25-1.00  &  -0.029  &  81  \\
\hline
\vspace{60pt}
\end{tabular}
&
\parbox{0.5\linewidth}{\includegraphics[scale=0.15]{results/_archive/figures/lpoly_HHI.png}}
\end{tabular}

\bigskip \bigskip
\raggedright

{\hyperlink{summary}{\beamerbutton{back}}}

\end{frame}

%%%

\end{document}

%%%

\begin{frame}{title}
\begin{itemize}

\item x

\end{itemize}
\end{frame}



%%% EXTRAS



%%%

\begin{frame}{Persistence of referrals by specialist HHI}
\centering
\small 

\begin{tabular}{cc}
Mean MFX by HHI &
Local Linear Regression \\
\\
\begin{tabular}{lcc}
\hline
HHI  &  MFX  &  \# mkts  \\
\hline
0.0-0.1  &   0.765 &  65  \\
0.1-0.2  &  0.693  &  91  \\
0.2-0.4  &  0.659  &  88  \\
0.4-1.0  &  0.625  &  32  \\
\hline
\vspace{48pt}
\end{tabular}
&
\parbox{0.5\linewidth}{\includegraphics[scale=0.25]{results/_archive/figures/lpoly_inertia_HHI.png}}
\end{tabular}

shop less when you have more choice

\end{frame}
