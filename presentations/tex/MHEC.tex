%%% SETUP
\documentclass[slides, smaller, aspectratio=169]{beamer}

\setbeameroption{hide notes}
%\setbeameroption{show notes}
%\setbeamertemplate{note page}[plain]
%\setbeameroption{show notes on second screen}

% Theme:

\usetheme{Malmoe}
\usecolortheme[RGB={101, 56, 25}]{structure}

\setbeamertemplate{itemize items}[default]
\setbeamertemplate{itemize subitem}[circle]

\setbeamertemplate{navigation symbols}{}

% Packages:
\usepackage{amssymb,amsmath,amsfonts,amsthm}
\usepackage{graphicx, tikz}
\usepackage{setspace}
\usepackage{xkeyval}
\usepackage{color, colortbl}

% Commands:
\newcommand{\E}{\text{E}}
\newcommand{\V}{\mathrm{V}}
\newcommand{\dd}{\mathrm{d}}
\newcommand{\beq}{\begin{equation}}
\newcommand{\eeq}{\end{equation}}

\newcommand{\bit}{\begin{itemize}}
\newcommand{\eit}{\end{itemize}}

\newcommand{\benu}{\begin{enumerate}}
\newcommand{\eenu}{\end{enumerate}}

\definecolor{themebrown}{RGB}{101, 56, 25}

% Title:

\title[Physician Referrals \hspace{4cm} \insertframenumber]{Learning and Efficiency in the Market for Physician Referrals}
\author[McCarthy \and Richards-Shubik]%
{Ian McCarthy\inst{1} \and Seth Richards-Shubik\inst{2}}
\institute[]{
  \inst{1} Emory University and NBER
  \and
  \inst{2} Johns Hopkins University and NBER
}
\date[]{Midwest Health Economics Conference \\ September 2024}

\begin{document}

\frame{\titlepage}


%%%%%%%%%%%%%%%%%%%%%%%%%%%%%%%%%%%%
%%% DR. DEATH INTRO
%%%%%%%%%%%%%%%%%%%%%%%%%%%%%%%%%%%%

\begin{frame}
\begin{center}
\LARGE

\color[RGB]{89, 38, 11}
A (SAD/TERRIFYING) STORY

\end{center}
\end{frame}

\begin{frame}{}
\begin{center}
\only<1>{
    \includegraphics[scale=0.3]{presentations/images/dr-death-cover.png}
}
\end{center}
\only<2>{
\begin{itemize}
    \item Severely injured or killed 33 of 38 patients over 2+ years
    \item First doctor to be sentenced to life in prison for medical malpractice
\end{itemize}
    \invisible{
        \center{
            \includegraphics[scale=0.3]{presentations/images/mugshots.png}
        }
    }
}

\only<3>{
\begin{itemize}
    \item Severely injured or killed 33 of 38 patients over 2+ years
    \item First doctor to be sentenced to life in prison for medical malpractice
\end{itemize}
        \center{
            \includegraphics[scale=0.3]{presentations/images/mugshots.png}
        }
}

\end{frame}


%%%%%%%%%%%%%%%%%%%%%%%%%%%%%%%%%%%%
%%% ECONOMIC INTRO
\begin{frame}
\begin{center}
\LARGE

\color[RGB]{89, 38, 11}
ECONOMIC MOTIVATION

\end{center}
\end{frame}


\begin{frame}{Physician referrals}
\begin{itemize}

    \item Service market where professionals with general skills (e.g., primary care) direct consumers to professionals with specialized skills
    \bit
        \item Substantial heterogeneity in spending and quality across specialists
        \item Opportunity for learning in referrals
    \eit
    \bigskip 
    \pause

    \item Referrals more prevalent over time
    % Data from: National Ambulatory Medical Care Survey (and hospital version), large nationally representative survey of physician offices, run by NCHS, with 20,000 to 35,000 observations per year
    \medskip
    \pause

    \footnotesize
    \begin{tabular}{lccccc}
    \hline
        & \multicolumn{2}{c}{Rate per Visit} &   &  \multicolumn{2}{c}{Annual Total} \\
        &   1999    &   2009    &   &   1999    &   2009 \\
    \hline
    Referrals from primary care visits  &   5.8\%   &   9.9\%   &   &   22M &   51M \\
    Referrals among specialists         &   2.9\%   &   7.3\%   &   &   11M &   38M \\
    Patient self-referrals to specialists & 6.0\%   &   2.8\%   &   &   51M &   31M \\
    % self-referral = any new patient visit that was marked as being not referred
    \hline
    \end{tabular}
    
    \smallskip
    \parbox{\linewidth}{\footnotesize
        (Barnett et al.~2012)
    }
    \normalsize
    \smallskip
    \pause
    
    \item Referring physicians have large influence on choice of specialist
    \smallskip
    \parbox{\linewidth}{\footnotesize
        (e.g., Freedman, Kouri, West, and Keating, \emph{JAMA Oncology} 2015; Chernew, Cooper, Hallock, and Scott Morton, \emph{JHlthEcon} 2021)
    }
    \smallskip
    \pause 

    \item Referring physicians say patient outcomes and experiences matter
    \smallskip
    \parbox{\linewidth}{\footnotesize
        (e.g., Schneider and Epstein, \emph{NEJM} 1996; Barnett, Keating, Christakis, O'Malley, and Landon, \emph{JGenIntMed} 2012)
    }

\end{itemize}
\end{frame}


%%%%%%%%%%%%%%%%%%%%%%%%%%%%%%%%%%%%
%%% OUR PAPER AND CONTRIBUTION

\begin{frame}{This paper}
\begin{itemize}


    \item We develop a structural learning model for referral decisions and apply it to a specific area of health care with a well defined network of referring physicians and specialists

    \medskip
    \pause

    \item Model also accounts for
    \bit
        \item Habit persistence (i.e., preference to refer to familiar specialists)
        \item Congestion / capacity constraints from specialists
        \item Specialist-level unobserved factors (fixed effects)
    \eit

    \medskip 
    \pause
    \label{summary}


    \item Use model to simulate ability of market to better allocate patients to specialists if learning is improved
    \ \hyperlink{literature}{\beamerbutton{contribution}}
    \bit
        \item PCPs know specialist quality perfectly
        \item ...and other heuristics are shut down
    
    \eit
    \medskip
    \pause

    \item Today's results: 
    \bit
        \item Referral patterns, heterogeneity in quality and cost
        \item Design-based evidence of response to quality
        \item Structural model of (myopic) learning with counterfactuals
    \eit 

\end{itemize}
\end{frame}

%%%%%%%%%%%%%%%%%%%%%%%%%%%%%%%%%%%%




%%%%%%%%%%%%%%%%%%%%%%%%%%%%%%%%%%%%
% NEW SECTION SLIDE

\begin{frame}
\begin{center}
\LARGE

\color[RGB]{89, 38, 11}
CONTEXT \& DATA

\color{lightgray}
DESIGN-BASED EVIDENCE

\color{lightgray}
STRUCTURAL MODEL

\end{center}
\end{frame}

%%%%%%%%%%%%%%%%%%%%%%%%%%%%%%%%%%%%



%%%%%%%%%%%%%%%%%%%%%%%%%%%%%%%%%%%%
% CONTEXT

\begin{frame}{Application: joint replacement surgery}
\begin{itemize}

    \item About 500,000 hip replacements and 1,000,000 knee replacements in the US per year (and rapidly growing)
    % e.g., prediction: 780-925k hip and 1.5-2.4m knee in 2030; so a 50-100% increase by then
    % https://doi.org/10.3899/jrheum.170990
    \bit
        \item Expensive: \$23,000 mean/median (\$12,500 std.~dev.) 
        \item Some risk: 9.3\% rehospitalized, 0.6\% die from complications
        \item Medicare accounts for as much as 65\% of procedures
        % from our data
    \eit
    \bigskip
    \pause

    \item Patients often referred by their primary care physicians (PCPs) to orthopedic surgeons (specialists) for the procedure
    \bigskip 


\end{itemize}
\end{frame}


%%%%%%%%%%%%%%%%%%%%%%%%%%%%%%%%%%%%
% DATA

\begin{frame}{FFS Medicare claims, 2008--2018}
\label{data}
\begin{itemize}


    \item All fee-for-service Medicare claims for major joint replacements from 2008 to 2018: 4.5 million surgeries
    \bigskip
    \pause

    \item Referring physician (PCP) inferred from regular office visits over prior year (2 or more)
    \bit
        \item 87\% of surgeries are matched to a PCP
        \item Restrict to PCPs and orthopedic surgeons with sufficient volume 
        % (PCP: 20+ total referrals, specialist: 20+ surgeries per year)  
        % PCP: at least 20 total referrals over our study period and at least three consecutive years with observed referrals
        % specialist: perform at least 20 surgeries per year (captures xxx\% of all surgeries)
        \item 2,014,300 surgeries, 10,500 surgeons, 51,700 PCPs in final sample
        \ \hyperlink{sumstats}{\beamerbutton{summary statistics}}
    \eit
%    \bigskip 
%    \pause

%    \item Note -- the claims include a ``referring physician'' field, but it is highly unreliable: it is omitted on 34\% of the joint replacement claims, and it lists the \emph{same person} as the operating physician on 64\% of the claims where it is populated.

\end{itemize}
\end{frame}

%%%

\begin{frame}{Referral Concentration}
{Computed per PCP, using six years of data from 2013 to 2018}

\footnotesize

\begin{tabular}{cc}
    \includegraphics[scale=0.24]{results/figures/desc/NetworkSize_1_1_0.png} &
    \includegraphics[scale=0.24]{results/figures/desc/HighestShare_1_1_0.png} \\
    Mean = 9.6 & Mean = 0.33
\end{tabular}

\bigskip
\small

\emph{Average number of specialists per HRR: 83}

\emph{\ldots and within 75 miles or 90th percentile: 81}
% Most PCPs refer their patients to less than 1/4 of specialists in their market

\end{frame}

\begin{frame}{Heterogeneity in quality and cost}
{Interquartile ranges within markets of negative outcomes and episode spending}

\footnotesize

\begin{tabular}{cc}
    \includegraphics[scale=0.24]{results/figures/desc/Failure_IQR_1_1_0.png} &
    \includegraphics[scale=0.24]{results/figures/desc/Payment_IQR_1_1_0.png} \\
    Average IQR = 0.046 & Average IQR = \$7,932
\end{tabular}

\bigskip

\small

\emph{If patients could be reallocated from worse to better specialists \\
(75th to 25th percentiles of negative outcomes or episode spending), substantial improvements are possible for those patients.}

\end{frame}



%%%%%%%%%%%%%%%%%%%%%%%%%%%%%%%%%%%%
% NEW SECTION SLIDE

\begin{frame}

\vskip24pt

\begin{center}
\LARGE

\color{lightgray}
CONTEXT \& DATA

\color[RGB]{89, 38, 11}
DESIGN-BASED EVIDENCE

\color{lightgray}
STRUCTURAL MODEL

\end{center}

\end{frame}
%%%%%%%%%%%%%%%%%%%%%%%%%%%%%%%%%%%%


\begin{frame}[t]{General Idea \ \hyperlink{eventstudy}{\beamerbutton{details}}}
\label{es-general}
\begin{itemize}
    \item Construct balanced panel of PCP-specialist pairs across quarters
    \item Compare referrals to same specialist among two types of PCPs:
    \bit
        \item PCPs whose patient(s) experienced a failure
        \item PCPs whose patient(s) did not experience a failure
    \eit
    \item Estimate stacked DD (Cengiz \textit{et al.} 2019, QJE), stacked by the first, second, third, and fourth failure events per specialist
\end{itemize}

\end{frame}


\begin{frame}{Event study of specialist's first failure event}
{Effect on referrals per quarter from PCP observing the failure vs.~other PCPs}

\centering
\includegraphics[scale = 0.32]{results/figures/rf/EventStudy_Stacked_1_1_0.png}

\footnotesize
Dependent var.~mean = 0.194 (treated group before failure), relative reduction of 31\%

\end{frame}


\begin{frame}{Event study of specialist's first failure event}
{Alternative specification with specialist/quarter FEs}

\centering
\includegraphics[scale = 0.32]{results/figures/rf/EventStudy_FEq_Stacked_1_1_0.png}

\end{frame}



%%%%%%%%%%%%%%%%%%%%%%%%%%%%%%%%%%%%
% NEW SECTION SLIDE

\begin{frame}

\vskip24pt

\begin{center}
\LARGE

\color{lightgray}
CONTEXT \& DATA

\color{lightgray}
DESIGN-BASED EVIDENCE

\color[RGB]{89, 38, 11}
STRUCTURAL MODEL

\bigskip

\small

\emph{PCPs refer a sequence of patients to a set of specialists, and learn about the quality of those specialists from the outcomes of their patients.}

\end{center}
\end{frame}

%%%%%%%%%%%%%%%%%%%%%%%%%%%%%%%%%%%%
% MODEL

\begin{frame}{Model specification}
\begin{enumerate}

    \item Each period $t$, the PCP $i$ sends a patient (also $t$) to a specialist $j$ from a fixed choice set $J$
    \bigskip
    \pause

    \item Specialist treats the patient
    \bit
        \item binary outcomes: $Y_{ijt} = 1$ (success) or 0 (failure)
        \item specialist quality: $q_j \equiv \Pr(Y_{ijt} = 1)$,
        \emph{unknown to PCP}
    \eit 
    \bigskip 
    \pause
    
    \item PCP's realized utility:    
    \color[RGB]{89, 38, 11}
    $$U_{ijt} \equiv \alpha Y_{ijt} + u(x_{ijt}) + f(e_{ijt}) + c(n_{jt}, z_j) + \xi_j + \epsilon_{ijt},$$
    \color{black}

    \only<3> {
    \vspace{-5mm}
    \bit
        \item $\alpha$ -- weight on patient outcomes (e.g., altruism)
        \item $u(x_{ijt})$ -- patient-specific factors (e.g., distance)
        \item $f(e_{ijt})$ -- prior relationship with specialist (familiarity)
        \item $c(n_{jt}, z_j)$ -- specialist current patient volume (congestion effect)
        % (as in Richards-Shubik \emph{et al.}~2022)
        \item $\xi_{j}$ -- other demand factors (unobserved to econometrician)
        \item $\epsilon_{ijt}$ -- idiosyncratic shock
    \eit
    \vspace{-10mm}
    }
    \pause

    \vspace{-2mm}
    \item[$\blacktriangleright$] Model captures key features related to the potential to improve market allocations:
    \bit
        \item learning about quality
        \item taste for familiarity        
        \item capacity constraints
    \eit

    \only<1-2,4>{
    \vspace{7mm}
    }
    
\end{enumerate}
\end{frame}


\begin{frame}{Learning and referral decisions}
\begin{itemize}

    \item PCP beliefs about specialist quality: $q_j \sim \text{Beta}$
    \smallskip
    \pause

    \item Update based on patient outcomes:
    % mention beta distribution, a natural and tractable assumption
    \begin{align*}
        m_{ijt} \equiv \E[q_j | (D_{ijs}, Y_{ijs})_{s=1}^{t-1}]
         & = \frac{ a_0 +  \sum_{s=1}^{t-1} Y_{ijs} }{ a_0 + b_0 + \sum_{s=1}^{t-1} D_{ijs} } \\
         & = \frac{ a_0 +  y_{ijt}}{ a_0 + b_0 + e_{ijt} }
    \end{align*}
    \bit    
        \item $a_0, b_0$ -- parameters of prior beliefs \\
        \ \ prior mean $= a_0 / (a_0 + b_0)$ \\
        \ \ prior ``strength'' $= a_0 + b_0$
        \item $e_{ijt} = \sum_{s=1}^{t-1} D_{ijs}$ -- number of past patients referred to $j$
        \item $y_{ijt} = \sum_{s=1}^{t-1} Y_{ijs}$ -- successes among those patients
    \eit

\end{itemize}
\end{frame}

\begin{frame}{Myopic Learning}
\begin{itemize}

    \item PCP's realized utility:    
    \color[RGB]{89, 38, 11}
    $$U_{ijt} \equiv \alpha Y_{ijt} + u(x_{ijt}) + f(e_{ijt}) + c(n_{jt}, z_j) + \xi_j + \epsilon_{ijt},$$
    \color{black}
    \medskip
    \pause

    \item Referral based on maximum expected utility:
    \begin{align*}
        \max_{j \in J} \ \text{E} \left[ U_{ijt} | \dots \right] =
\max_{j \in J} \ \ & \alpha m_{ijt} + \pi x_{ijt} \\
& + \sum_{p}\beta_{p} I(e_{p-1}\leq e_{ijt} < e_{p}) \\ 
& + \underbrace{c(n_{j}, z_j) + \xi_j}_{\delta_j} + \epsilon_{ijt}
    \end{align*}


\end{itemize}
\end{frame}


\begin{frame}{Myopic Learning}

\textbf{Expected maximum utility:}
\begin{equation*}
    \text{E} \left[ U_{ijt} | \dots \right] =
    \alpha m_{ijt}
    + \pi x_{ijt} + \sum_{p}\beta_{p} I(e_{p-1}\leq e_{ijt} < e_{p}) + \underbrace{c(n_{j}, z_j) + \xi_j}_{\delta_j} + \epsilon_{ijt} .
\end{equation*}
\vskip100pt

\only<2>{
\vskip-100pt
\begin{itemize}
    \item For computational ease, we parameterize priors:
    \begin{equation*}
         \rho \equiv \frac{a_0}{a_0 + b_0}
        \ \ \text{ and } \ \ 
        \eta \equiv a_0 + b_0
    \end{equation*}
    \medskip
    \pause 
    
    \item Hence, mean of the current beliefs about the quality of specialist $j$ in period $t$ is $$m_{ijt} = \frac{\rho \eta + y_{ijt}}{\eta + e_{ijt}}$$
\end{itemize}
}


\only<3>{
\vskip-100pt
\textbf{Final parameterization:}
\begin{equation*}
    \text{E} \left[ U_{ijt} | \dots \right] =
    \alpha \frac{\rho \eta + y_{ijt}}{\eta + e_{ijt}}
    + \pi x_{ijt} + \sum_{p}\beta_{p} I(e_{p-1}\leq e_{ijt} < e_{p}) + \underbrace{c(n_{j}, z_j) + \xi_j}_{\delta_j} + \epsilon_{ijt} .
\end{equation*}
}

\end{frame}




%%%%%%%%%%%%%%%%%%%%%%%%%%%%%%%%%%%%
% IDENTIFICATION
\begin{frame}[t]{Identification of the myopic model} 
% NOTE: This formal analysis is new.  Prior papers (Dickstein 2021, Gong 2018) have thoughtful discussions but do not provide formal analysis.

\vskip-12pt

\begin{equation*}
    \text{E} \left[ U_{ijt} | \dots \right] =
    \alpha \frac{\rho \eta + y_{ijt}}{\eta + e_{ijt}}
    + \pi x_{ijt} + \sum_{p}\beta_{p} I(e_{p-1}\leq e_{ijt} < e_{p}) + \underbrace{c(n_{j}, z_j) + \xi_j}_{\delta_j} + \epsilon_{ijt} .
\end{equation*}

\vskip-12pt

    \only<1>{
    \bigskip
    \bigskip
    \emph{Identifying variation comes from differences in histories of patient outcomes across PCPs referring to the same specialists}
    }
    \pause


\begin{itemize}
    
    \item Specialist fixed effects ($\delta_{j}$): many patients relative to specialists
    \pause

    \item Patient $\times$ alternative characteristics $(\pi)$: patients arrive exogenously
    \pause

    \label{utility specification}
    \item Congestion effect ($\gamma$): identified using instrument for referrals from other PCPs (Richards-Shubik \emph{et al.}~2022) 
    \ \hyperlink{congestion}{\beamerbutton{congestion}}
    \pause

    \label{identification}
    \item Altruism ($\alpha$): variation in success rates among specialists to whom a PCP has sent many patients in the past
    \ \hyperlink{identification details}{\beamerbutton{algebra}}
    \pause

    \item Familiarity ($\beta_{p}$): variation in counts of past referrals for same observed quality
    \pause

    \item Prior beliefs
    \bit
        \item Mean ($\rho=\frac{a_0}{a_0 + b_0}$): average success rate in market (rat'l expectations)
        \smallskip
        \item Strength ($\eta=a_0 + b_0$): how the marginal effect of success ($\sum_{s=1}^{t-1} Y_{ijs}$) changes with experience ($\sum_{s=1}^{t-1} D_{ijs}$) \\
        % i.e., their interaction
        \item But strength seems to be poorly identified, so we fix it at $\eta \in \{1,5\}$
    \eit 

\end{itemize}
\end{frame}

%%%%%%%%%%%%%%%%%%%%%%%%%%%%%%%%%%%%
% Estimation
\begin{frame}[t]{Estimation} 

\vskip-12pt

\begin{equation*}
    \text{E} \left[ U_{ijt} | \dots \right] =
    \alpha \frac{\rho \eta + y_{ijt}}{\eta + e_{ijt}}
    + \pi x_{ijt} + \sum_{p}\beta_{p} I(e_{p-1}\leq e_{ijt} < e_{p}) + \underbrace{c(n_{j}, z_j) + \xi_j}_{\delta_j} + \epsilon_{ijt} .
\end{equation*}

\textbf{Standard MNL with two caveats:}
\begin{enumerate}
    \item Separate congestion, $c(n_{j}, z_{j})$, from $\xi_{j}$
    \item Successful patient outcomes $(Y_{ijt}=1)$ are weakly preferred by PCPs, i.e., $\alpha \geq 0$
\end{enumerate}
\end{frame}

\begin{frame}[t]{Estimation: Congestion} 

\begin{itemize}
    \item Simplify congestion term such that $c(n_{j}, z_j) \equiv \gamma n_j$ (count of specialist's patients)
    \medskip
    \pause
    
    \item Standard MNL where $\gamma n_{j}$ is absorbed into the composite fixed effect, $\delta_j$
    \medskip
    \pause 
    
    \item $\gamma$ estimated with a linear regression defined by the identity $\delta_j \equiv \gamma n_j + \xi_j$ using 2SLS
    \bit
        \item Instrument is $\hat n_j$ from MNL with only distance as the independent variable. 
        \item \textit{Identifying variation:} Distances between \emph{all other patients} and specialist $j$ \\
    \eit
\end{itemize}

\medskip
\pause
\footnotesize

$\hat n_j$ is generated using the distances, $x_{kjs}$, from all PCPs $k$ and patients $s$ in the market, while only the current patient's distance, $x_{ijt}$, affects the individual referral decision

\end{frame}

\begin{frame}[t]{Estimation: Preference for Good Outcomes} 

\begin{itemize}
    \item Exponential transformation (i.e., $\alpha = \text{e}^{\tilde \alpha}$)
    \medskip
    \pause 
    
    \item Follow algorithm proposed in Fader \textit{et al.} 1992, Marketing Science:
    \bit
        \item initial value, $\tilde \alpha_{0}=\ln (\alpha_0)$
        \item construct $\tilde{m}_{ijt,0}=\text{e}^{\tilde \alpha_{0}} \times m_{ijt}$
        \item estimate the coefficient on this transformed variable, $\widehat{\Delta}_{0}$
        \item update with $\tilde \alpha_{1} = \tilde \alpha_{0} + (\widehat{\Delta}_{0} - 1)$
        \item continue until $\tilde \alpha$ converges, which we denote as $\tilde \alpha^{*}$
    \eit
    \medskip
    \pause 
    
    \item Final estimate for $\alpha$ follows from the transformation of $\tilde \alpha^{*}$, with $\hat{\alpha} = \text{e}^{\tilde \alpha^{*}}$
\end{itemize}
\end{frame}


\begin{frame}[t]{Estimation} 

\vskip-12pt

\begin{equation*}
    \text{E} \left[ U_{ijt} | \dots \right] =
    \alpha \frac{\rho \eta + y_{ijt}}{\eta + e_{ijt}}
    + \pi x_{ijt} + \sum_{p}\beta_{p} I(e_{p-1}\leq e_{ijt} < e_{p}) + \underbrace{c(n_{j}, z_j) + \xi_j}_{\delta_j} + \epsilon_{ijt} .
\end{equation*}

\textbf{Standard MNL with two caveats:}
\begin{enumerate}
    \item Separate congestion, $c(n_{j}, z_{j})$, from $\xi_{j}$
    \item Successful patient outcomes $(Y_{ijt}=1)$ are weakly preferred by PCPs, i.e., $\alpha \geq 0$
\end{enumerate}
\medskip

\textbf{Estimate 612 choice models:}
\begin{itemize}
    \item Separately for all 306 HRRs
    \item Separately for $\eta=1$ and $\eta=5$
\end{itemize}

\end{frame}

\begin{frame}[t]{Parameter Estimates} 

\footnotesize
\begin{center}
\begin{tabular}{lrrrrrrr}
 & & & \multicolumn{5}{c}{Percentile} \\
 \cline{4-8}
Parameter & Mean & (SD/SE) & 10th & 25th & 50th & 75th & 90th \\ 
\hline
\\
\multicolumn{8}{l}{$\alpha$ (utility weight on outcome)} \\ 
\ \ ($\eta=1$) & 0.304 & (0.275) & 0.030 & 0.092 & 0.221 & 0.429 & 0.706 \\ 
\ \ ($\eta=5$) & 0.712 & (0.651) & 0.090 & 0.228 & 0.522 & 0.979 & 1.631 \\ 
\\
\multicolumn{8}{l}{$\pi$ (utility weight on distance)} \\ 
\ \ ($\eta=1$) & -0.0706 & (0.0060) & -0.1011 & -0.0862 & -0.0719 & -0.0541 & -0.0430 \\ 
\ \ ($\eta=5$) & -0.0702 & (0.0061) & -0.1021 & -0.0863 & -0.0722 & -0.0553 & -0.0457 \\ 
\\ 
\multicolumn{8}{l}{$\gamma$ (congestion effect, per 100 patients)} \\ 
\ \ ($\eta=1$) & -0.021 & (0.0086) \\ 
\ \ ($\eta=5$) & -0.021 & (0.0089) \\ 
\hline
\end{tabular}
\end{center}

\end{frame}


\begin{frame}[t, noframenumbering]{Parameter Estimates} 

\footnotesize
\begin{center}
\begin{tabular}{lrrrrrrr}
 & & & \multicolumn{5}{c}{Percentile} \\
 \cline{4-8}
Parameter & Mean & (SD/SE) & 10th & 25th & 50th & 75th & 90th \\ 
\hline
\\ 
\multicolumn{8}{l}{$\alpha$ (utility weight on outcome)} \\ 
\rowcolor{red} \ \ ($\eta=1$) & 0.304 & (0.275) & 0.030 & 0.092 & 0.221 & 0.429 & 0.706 \\ 
\rowcolor{red} \ \ ($\eta=5$) & 0.712 & (0.651) & 0.090 & 0.228 & 0.522 & 0.979 & 1.631 \\ 
\\
\multicolumn{8}{l}{$\pi$ (utility weight on distance)} \\ 
\ \ ($\eta=1$) & -0.0706 & (0.0060) & -0.1011 & -0.0862 & -0.0719 & -0.0541 & -0.0430 \\ 
\ \ ($\eta=5$) & -0.0702 & (0.0061) & -0.1021 & -0.0863 & -0.0722 & -0.0553 & -0.0457 \\ 
\\
\multicolumn{8}{l}{$\gamma$ (congestion effect, per 100 patients)} \\ 
\ \ ($\eta=1$) & -0.021 & (0.0086) \\ 
\ \ ($\eta=5$) & -0.021 & (0.0089) \\ 
\hline
\end{tabular}
\end{center}

\bigskip
\textbf{Takeaways:}
\begin{itemize}
    \item PCPs place positive weight on patient outcomes
    \item Weight is increasing in $\eta$, natural because $\alpha$ is divided by $\eta + e$ to form a composite ``coefficient'' on $y$ in utility specification
\end{itemize}

\end{frame}

\begin{frame}[t, noframenumbering]{Parameter Estimates} 

\footnotesize
\begin{center}
\begin{tabular}{lrrrrrrr}
 & & & \multicolumn{5}{c}{Percentile} \\
 \cline{4-8}
Parameter & Mean & (SD/SE) & 10th & 25th & 50th & 75th & 90th \\ 
\hline
\\ 
\multicolumn{8}{l}{$\alpha$ (utility weight on outcome)} \\ 
\ \ ($\eta=1$) & 0.304 & (0.275) & 0.030 & 0.092 & 0.221 & 0.429 & 0.706 \\ 
\ \ ($\eta=5$) & 0.712 & (0.651) & 0.090 & 0.228 & 0.522 & 0.979 & 1.631 \\ 
\\
\multicolumn{8}{l}{$\pi$ (utility weight on distance)} \\ 
\rowcolor{red} \ \ ($\eta=1$) & -0.0706 & (0.0060) & -0.1011 & -0.0862 & -0.0719 & -0.0541 & -0.0430 \\ 
\rowcolor{red} \ \ ($\eta=5$) & -0.0702 & (0.0061) & -0.1021 & -0.0863 & -0.0722 & -0.0553 & -0.0457 \\ 
\\
\multicolumn{8}{l}{$\gamma$ (congestion effect, per 100 patients)} \\ 
\ \ ($\eta=1$) & -0.021 & (0.0086) \\ 
\ \ ($\eta=5$) & -0.021 & (0.0089) \\ 
\hline
\end{tabular}
\end{center}

\bigskip
\textbf{Takeaways:}
\begin{itemize}
    \item Preference for closer hospitals/specialists
    \item Stable across $\eta$
\end{itemize}

\end{frame}

\begin{frame}[t, noframenumbering]{Parameter Estimates} 

\footnotesize
\begin{center}
\begin{tabular}{lrrrrrrr}
 & & & \multicolumn{5}{c}{Percentile} \\
 \cline{4-8}
Parameter & Mean & (SD/SE) & 10th & 25th & 50th & 75th & 90th \\ 
\hline
\\ 
\multicolumn{8}{l}{$\alpha$ (utility weight on outcome)} \\ 
\ \ ($\eta=1$) & 0.304 & (0.275) & 0.030 & 0.092 & 0.221 & 0.429 & 0.706 \\ 
\ \ ($\eta=5$) & 0.712 & (0.651) & 0.090 & 0.228 & 0.522 & 0.979 & 1.631 \\ 
\\
\multicolumn{8}{l}{$\pi$ (utility weight on distance)} \\ 
\ \ ($\eta=1$) & -0.0706 & (0.0060) & -0.1011 & -0.0862 & -0.0719 & -0.0541 & -0.0430 \\ 
\ \ ($\eta=5$) & -0.0702 & (0.0061) & -0.1021 & -0.0863 & -0.0722 & -0.0553 & -0.0457 \\ 
\\
\multicolumn{8}{l}{$\gamma$ (congestion effect, per 100 patients)} \\ 
\rowcolor{red} \ \ ($\eta=1$) & -0.021 & (0.0086) \\ 
\rowcolor{red} \ \ ($\eta=5$) & -0.021 & (0.0089) \\ 
\hline
\end{tabular}
\end{center}

\bigskip
\textbf{Takeaways:}
\begin{itemize}
    \item Some evidence of congestion, albeit relatively small
    \item An increase of 100 patients (nearly 2/3rds of a standard deviation) implies a 1.6\% relative reduction in referral probabilities
\end{itemize}

\end{frame}



\begin{frame}[t]{Parameter Estimates} 

\footnotesize
\begin{center}
\begin{tabular}{lrrrrrrr}
 & \multicolumn{7}{c}{$e_{ijt}$ interval} \\
 & [1,5)   & [5,10)  & [10,15) & [15,20) & [25,30) & [30,35) & [35,40) \\ 
\hline
\multicolumn{8}{l}{$\beta_{p}$ (familiarity)} \\ 
\ \ ($\eta=1$) & 1.291   & 2.123   & 2.423   & 2.972   & 3.151   & 3.552   & 3.671   \\ 
               & (0.097) & (0.137) & (0.196) & (0.213) & (0.259) & (0.264) & (0.333) \\ 
\ \ ($\eta=5$) & 1.291   & 2.122   & 2.421   & 2.971   & 3.149   & 3.550   & 3.669   \\ 
               & (0.097) & (0.136) & (0.196) & (0.213) & (0.259) & (0.264) & (0.333) \\ 
\end{tabular}
\end{center}

\bigskip
\textbf{Takeaways:}
\begin{itemize}
    \item Relatively large and increasing in range of $e$
    \item Estimates similar across $\eta$
\end{itemize}
\end{frame}


\begin{frame}[t]{Partial Effect of Failures} 
\begin{center}
\begin{tabular}{cc}
    \includegraphics[scale=0.22]{results/figures/myopic-timevary/Mean_Partial_Effect_Failure_eta1.png} &
    \includegraphics[scale=0.22]{results/figures/myopic-timevary/Mean_Partial_Effect_Failure_eta5.png} \\
\end{tabular}
\end{center}

\only<2>{
\medskip
\begin{itemize}
    \item Many markets with effectively no response to specialist failures
    \item Conditional on some response, mean reduction of 3-4\% in referral probability
    \item Small but meaningful over the course of several referrals: translates to 3 weeks worth of surgeries per year for an average specialist (among our sample of patients).
\end{itemize}
}

\end{frame}


\begin{frame}[t]{Counterfactual: Full Information} 
\begin{center}
\includegraphics[scale=0.31]{results/figures/myopic-timevary/Mean_Effect_Full_eta1.png}
\end{center}

\only<2>{
\begin{itemize}
    \item Small reallocation (less than 5\% in almost all markets)
    \item Subsequent improvements in health outcomes of around 0.5\%
\end{itemize}
}
\end{frame}

\begin{frame}[t]{Counterfactual: Full Information} 
\begin{center}
\includegraphics[scale=0.31]{results/figures/myopic-timevary/Mean_Health_Effect_Full_eta1.png}
\end{center}

\begin{itemize}
    \item \textit{Mean} health effects small (0.5\%), but meaningful in some markets ($>$1\% or 2\%)
    \item Moving from 90\% success rate to 91\% or 92\%
    \item Collectively 10-20k fewer complications/readmissions
\end{itemize}

\end{frame}


\begin{frame}[t]{Counterfactual: Full Information without Familiarity} 
\begin{center}
\includegraphics[scale=0.31]{results/figures/myopic-timevary/Mean_Effect_FullFam_eta1.png}
\end{center}

\only<2>{
\begin{itemize}
    \item Much larger reallocation, closer to 30\%
    \item Health effects remain relatively small, with increase in the \textit{ex ante} probability of a successful surgery of 0.5\% on average
\end{itemize}
}
\end{frame}



\begin{frame}{Summary} \label{summary}
\begin{itemize}
    \item Substantial quality and efficiency gains possible if referrals can be reallocated \emph{within} geographic areas

    \bigskip
    \item Structural learning model can quantify losses due to uncertainty and project gains from possible reallocation
    \bit
        \item While accounting for capacity constraints and other barriers
    \eit
    \bigskip

    \item Using variation in patient outcomes across PCPs referring patients to the same specialists, we find evidence of learning in some geographic markets

    \bigskip
    \item Full information could have large effects in settings where familiarity plays less of a role
    
\end{itemize}
\end{frame}


\begin{frame}[noframenumbering]
\begin{center}
\LARGE
\color[RGB]{89, 38, 11}
THANK YOU!

\medskip

\color{black}
\small
\textbf{Ian McCarthy} \\
Emory University \& NBER \\
\emph{email:} ian.mccarthy@emory.edu \\
\emph{website:} ianmccarthyecon.com
\end{center}
\end{frame}


%%%%%%%%%%%%%%%%%%%%%%%%%%%%%%%%%%%%
% APPENDIX SLIDES
%%%%%%%%%%%%%%%%%%%%%%%%%%%%%%%%%%%%

\begin{frame}[noframenumbering]
\begin{center}
\LARGE
\color[RGB]{89, 38, 11}
APPENDIX SLIDES
\end{center}
\end{frame}

%%%

\begin{frame}{Related literature \hyperlink{summary}{\beamerbutton{back}}}
\label{literature}
\begin{itemize}

    \item Structural learning models have been widely applied to study product choice and related problems (Ching, Erdem, and Keane 2013)
    \bigskip
    \pause

    \item Many studies in economics and marketing have applied Bayesian learning models to prescribing decisions by individual physicians
    % Focusing on examples in economics that use forward-looking models:
    \bit
        \item New pharmaceutical drugs \\
        \parbox{\linewidth}{\footnotesize (e.g., Ching 2010; Ferreyra and Kosenok 2011)}
        \item Patient-specific drug matches \\
        \parbox{\linewidth}{\footnotesize (e.g., Crawford and Shum 2005; Dickstein 2021)}
        \item Gong (2018) considers a new surgical procedure
        \item Johnson (2021) and Sarsons (2023) each consider learning in referrals
    \eit

    % Typical countefactuals (why need structural models): perfect information
    % Gittins solution -- Dickstein, Gong (Ferreyra and Kosenok use a similar threshold approach)
    \bigskip 
    \pause

    \item Other factors affecting referral choice:
    % Not in the model I am presenting today, but could potentially be incorporated.
    \bit
        \item peers from medical training
        \item distance between offices
        \item specialist gender, homophily 
        % (we would have no variation b/c nearly all orthopedic surgeons are men)
        \item vertical integration
    \eit

\end{itemize}
\end{frame}


\begin{frame}{Key summary statistics \hyperlink{data}{\beamerbutton{back}}}
\label{sumstats}
\footnotesize
\centering

\begin{tabular}{llrrr}
&    & (Baseline)     & (Estimation) \\
&    & 2008-2012     & 2013-2018 & Overall\\
\hline
\multicolumn{4}{l}{\emph{Per patient (episode):}}    \\
& 90-day Readmission  &       0.081&       0.066&       0.072\\
&                     &     (0.272)&     (0.249)&     (0.259)\\
& 90-day Mortality    &       0.007&       0.005&       0.006\\
&                     &     (0.081)&     (0.069)&     (0.074)\\
& 90-day Complication &       0.107&       0.089&       0.097\\
&                     &     (0.309)&     (0.285)&     (0.296)\\
& Failure             &       0.109&       0.091&       0.099\\
&                     &     (0.312)&     (0.288)&     (0.298)\\
& Episode spending    &      23,240&      22,557&      22,839\\
&                     &    (12,717)&    (12,284)&     (12,469)\\[1pt]
\multicolumn{4}{l}{\emph{Per PCP, per year:}}    \\
& Total Referrals     &       3.809&       4.398&       4.134\\
&                    &     (2.940)&     (3.478)&     (3.261)\\
& Unique Specialists  &       2.666&       3.109&       2.910\\
&                    &     (1.580)&     (1.831)&     (1.737)\\[1pt]
\multicolumn{4}{l}{\emph{Per PCP-specialist pair, past five years:}}    \\
& Total Referrals     & - - -    &        4.445 & - - -\\
&                     &     &     (10.453)\\
& Failure Rate  & - - -    &       0.101 & - - -\\
&                     &     &     (0.201)\\
% Last panel does not count zeros...
\hline
\end{tabular}

\end{frame}

\begin{frame}[t]{Event study \ \hyperlink{es-general}{\beamerbutton{back}}}
\label{eventstudy}
{Effect on referrals per quarter from PCP ``observing'' the failure vs.~other PCPs}
\begin{itemize}

    \item Create a quarterly panel of all PCP-specialist pairs with at least one referral between them (at any point)
    \item Capture failure events:
    \bit
        \item Failures for specialist $j$ denoted $f=1,...,F_{j}$
        \item Quarter of failures denoted $q_{j}(f)$, so that $q_{j}(1)$ denotes the quarter of first failure for specialist $j$, 
        \item $\underline{q}$ and $\overline{q}$ denote the first and last quarter of the analysis, respectively.
    \eit

    \item Find all PCPs who ever refer to specialist $j$ in relevant quarters, split into groups $k\in {0,1}$ based on failures
    \item Quarterly patients referred to specialist $j$ from PCP type $k$, denoted $\bar{r}_{jkt}$.
\end{itemize}
\end{frame}

\begin{frame}[t]{Event study \ \hyperlink{es-general}{\beamerbutton{back}}}
{Effect on referrals per quarter from PCP ``observing'' the failure vs.~other PCPs}

With this notation and sample construction, we then estimate by OLS the following event study specification:
\begin{equation*}
  \bar{r}_{jkt} = \gamma_{j} + \gamma_{t} + \delta I(k=1) + \sum_{\substack{\tau=-9 \\ \tau \neq -1}}^{9} \lambda_{\tau} I(k=1, t=\tau) + \varepsilon_{jkt},
\end{equation*}

\end{frame}


\begin{frame}{Congestion effect}
{(Richards-Shubik, Roberts, and Donohue 2022)} \label{congestion}
\begin{itemize}

\item Based on Bayer and Timmins (2007) spatial equilibrium model:
spillovers among consumers who choose the same ``location''

\item Approach is similar to BLP demand estimation,
but congestion effect takes the place of a price
\bigskip

\item Estimate multinomial logit with specialist-time fixed effects ($\delta_{jt}$) 
\[
\alpha m_{ijt} + f(e_{ijt}) + u(x_{ijt}) + \underbrace{ c(n_{jt}, z_j) + \xi_j }_{\delta_{jt}} + \epsilon_{ijt}
\]

\item Recover congestion effect using regression of estimated fixed effects:
\[
\hat \delta_{jt} = c(n_{jt}, z_j) + \xi_j + \eta_{jt}
\]
\vskip-4pt
\bit
    \item instrument for $n_{jt}$ is distances from \emph{other} patients to specialist $j$
    \item e.g., 2SLS estimation if $c$ is linear in parameters
\eit
\medskip

\hyperlink{utility specification}{\beamerbutton{back}}

\end{itemize}
\end{frame}

%%%

\begin{frame}[noframenumbering, t]{Identification of the myopic model}
\label{identification details}

\vskip-12pt

$$\alpha m_{ijt} + f(e_{ijt}) + u(x_{ijt}) + c(n_{jt}, z_j) + \xi_j + \epsilon_{ijt}$$

\begin{itemize}

\item Altruism parameter ($\alpha$): marginal effect of success rate, in limit as $e \rightarrow \infty$ (``identification at infinity'')
\[
\alpha m_{ijt} 
= \alpha \frac{ a_0 +  \sum_{s=1}^{t-1} Y_{ijs} }{ a_0 + b_0 + \sum_{s=1}^{t-1} D_{ijs} }
= \alpha \frac{ a_0/e_{ijt} + \bar y_{ijt} }{ (a_0 + b_0)/e_{ijt} + 1 } 
\]
where $\bar y_{ijt} \equiv \sum_{s=1}^{t-1} Y_{ijs} / \sum_{s=1}^{t-1} D_{ijs}$
and $e_{ijt} \equiv \sum_{s=1}^{t-1} D_{ijs}$
\vskip3pt
% Thus cases where a PCP has large past experience with two specialists, but different success rates with them, identify the altruism parameter.
\medskip

\item Strength of priors ($a_0 + b_0$):
how the marginal effect of the success rate changes with experience (i.e., interaction)
\[
\alpha m_{ijt} 
= \alpha \frac{ a_0 }{ (a_0 + b_0) + e_{ijt} } 
+ \alpha \frac{ e_{ijt} }{ (a_0 + b_0) + e_{ijt} } \times \bar y_{ijt}
\]
\vskip-8pt
\medskip

% IF TIME: Make appendix slide with this algebra
% \beamerbutton{algebra}

\item Prior mean ($\frac{a_0}{a_0 + b_0}$): estimate outside the model, using average success probability in each market (then have $a_0, b_0$ separately)
\medskip

\hyperlink{identification}{\beamerbutton{back}}

\end{itemize}
\end{frame}


\end{document}
